\documentclass[../notes.tex]{subfiles}

\pagestyle{main}
\renewcommand{\chaptermark}[1]{\markboth{\chaptername\ \thechapter\ (#1)}{}}
\setcounter{chapter}{3}

\begin{document}




\chapter{???}
\section{Cationic Polymerization of Olefins}
\begin{itemize}
    \item \marginnote{1/23:}Recap of anionic polymerizations.
    \begin{itemize}
        \item The double bond being attacked must be electron poor.
        \item You usually want monomers to have conjugation (this helps make them electron poor).
        \begin{itemize}
            \item Conjugation also helps stabilize the anion.
        \end{itemize}
        \item Your counterion and solvent are important.
        \begin{itemize}
            \item This is the biggest difference between radical and ionic polymerization.
            \item Even more important in today's discussion of cationic stuff.
        \end{itemize}
        \item The first living polymerization systems were anionic.
        \item Living systems:
        \begin{itemize}
            \item Can't have termination.
            \item $R_i\ggg R_t$.
            \item \textbf{Chain end fidelity}.
            \item PDI is close to 1.
            \item You can control the MW, in particular via conversion (first order kinetics) or ratio of monomer to initiator.
        \end{itemize}
        \item These are the key points we need to take away from last time.
    \end{itemize}
    \item There is an exam coming up pretty soon! When??
    \item \textbf{Chain end fidelity}: You have complete control of your ends.
    \begin{itemize}
        \item Helpful if you want to make block copolymers or something.
    \end{itemize}
    \item We now begin cationic polymerization.
    \item Cationic addition polymerization.
    \begin{itemize}
        \item The mechanism is an attack on the carbocation by the nucleophilic monomer (electron \emph{rich} double bond).
        \item We want the \emph{monomer} to be the most nucleophilic thing the carbocation sees, or we won't get the desired reactivity.
        \begin{itemize}
            \item Competition comes from the anion.
            \item \ce{Cl-} is quite nucleophilic, for instance; we want to cut down on this.
            \item Our monomer needs to be quite a bit more nucleophilic, or all we get is termination.
        \end{itemize}
    \end{itemize}
    \item Some monomers which polymerize cationically.
    \begin{itemize}
        \item Isobutylene and isoprene.
        \item Best ones are those that have a lone pair, e.g., alkoxy groups.
    \end{itemize}
    \item Characteristics of cationic polymerization.
    \begin{itemize}
        \item We have to stabilize this very high-energy carbocation.
        \item The obvious way is by delocalization, hence why styrene works well (even though it's not particularly nucleophilic, so we need a better ion in this case).
        \item We'll talk more about isobutylene later; it's a slightly different case.
        \item As we mentioned above, just like in the anionic case, counterion and solvent play an important role (actually a more important one).
        \item There are a wide variety of modes for both initiation and termianation.
        \begin{itemize}
            \item Any oxygen in a backbone is a decent nucleophile.
        \end{itemize}
        \item Low temperature can prevent side reactions.
        \begin{itemize}
            \item E.g., $-\SI{70}{\celsius}$ or even $-\SI{80}{\celsius}$.
            \item Lower temperature helps us not get over the higher activation barrier of certain side reactions.
        \end{itemize}
    \end{itemize}
    \item Regioselective electrophilic addition to substituted alkenes.
    \begin{itemize}
        \item Markovnikov's Rule (1870) sttes that the addition of \ce{HX} to an unsymmetrically substituted alkene results in the adduct in which \ce{H} has bonded to the carbon bearing the greater number of hydrogens.
        \begin{itemize}
            \item Initial protonation gives the most stable carbocation.
        \end{itemize}
        \item Stability of carbenium ions (carbocations).
        \begin{itemize}
            \item Caused by \textbf{hyperconjugation}.
            \item Example: A tertiary carbocation delocalizes the positive charge over \emph{nine} other hydrogens.
            \item Chemistry will go through \emph{gymnastics} to get us to the most stable tertiary carbocation. This is coming up in a couple slides!
        \end{itemize}
    \end{itemize}
    \item \textbf{Hyperconjugation}: Delocalization that invovles $\sigma$-bonded electrons rather than unshared/nonbonding- or $\pi$-electrons.
    \item What is required in a monomer and why?
    \begin{itemize}
        \item Monomers require two things.
        \begin{enumerate}
            \item The ability to form carbocations.
            \item The original monomer must be able to attack these newly formed cations.
        \end{enumerate}
        \item This then requires the monomers to be nucleophilic (positive nucleus seeking). The monomer must contain electron-rich double bonds.
    \end{itemize}
    \item Steric effects.
    \begin{itemize}
        \item Sterics can hinder the polymerization of 1,2-disubstituted olefins. However, if the substitutents are part of a cyclic structure, then the polymerization may occur.
        \begin{itemize}
            \item This is because ring strain provides an additional driving force!
        \end{itemize}
        \item Steric considerations can be used to explain the difference in polymerization of similar monomer units.
        \begin{itemize}
            \item Example: Norbornene has a less reactive double bond, but its sterics are far more favorable than with methylenenorbornene, even though the latter has an energetically favorable tertiary carbocation.
        \end{itemize}
    \end{itemize}
    \item The mechanism of cationic polymerization.
    \begin{itemize}
        \item There are \emph{many} ways by which this can take place.
    \end{itemize}
    \item Initiation.
    \begin{enumerate}
        \item Chemical methods.
        \begin{enumerate}
            \item Two-electron (heterolytic).
            \begin{enumerate}
                \item Br\o nsted (protic) acids (proton donor).
                \item Lewis acids.
            \end{enumerate}
        \end{enumerate}
    \end{enumerate}
    \item All of these are relevant, but the ones in red are the ones we'll focus on. \emph{More available in the slides.}
    \item Initiation generally consists of two processes.
    \begin{enumerate}
        \item Ion generation or priming.
        \begin{itemize}
            \item Br\o nsted acid: ...
            \item ...
            \item Aluminum\footnote{Historically, Webster's first dictionary misspelled the word as such; it was originally "aluminium," like "potassium" or "calcium." So that's why so many people pronounce it the Irish way.}: \ce{2 AlBr3 <=> AlBr2+ + AlBr4-}.
            \begin{itemize}
                \item Notice all the "big, hairy counterions."
            \end{itemize}
        \end{itemize}
        \item Cationation.
        \begin{itemize}
            \item ...
        \end{itemize}
    \end{enumerate}
    \item I. A. Br\o nsted (protic) acids.
    \begin{itemize}
        \item We generate \ce{H+} ions, and then the first \ce{H+} attacks a nucleophilic double bond, generating the first carbocation in the \textbf{cationation} step.
        \item Example of nucleophilicity:
        \begin{itemize}
            \item N-vinyl carbazole $>$ \ce{Cl-} $>$ \emph{iso}-butylene.
            \item Everything has to do with \emph{what's more nucleophilic}.
            \item We can polymerize N-vinyl carbazole using HCl, but not \emph{iso}-butylene!
            \item This is why the counterion is important.
        \end{itemize}
    \end{itemize}
    \item I. B. Lewis acids.
    \begin{itemize}
        \item This is why we use lewis acids more often --- they would bond to that chlorine and go on.
        \item Most of these are metal halides.
        \item Examples.
        \begin{itemize}
            \item \ce{AlCl3}, \ce{BF3}, \ce{SnCl4}, ...
        \end{itemize}
        \item There are organometallic derivatives.
        \begin{itemize}
            \item These are more soluble in the organic solvents we use, and hence better for that reason.
            \item Examples: \ce{RAlCl2}, \ce{R2AlCl}, \ce{R3Al}.
        \end{itemize}
        \item Most ofen, we need something in addition to the metal halide.
        \begin{itemize}
            \item Takeaway: We need some kind of \emph{donor} to give us these cationic species.
        \end{itemize}
        \item Example: If we add a touch of water to \ce{AlEt3}, then we can generate \ce{H+} and \ce{AlEt3OH-}.
        \item It is easy to separate the metal halides afterwards because we precipitate our polymer in a bad solvent for the polymer, filter out our polymer, and the catalyst stays in solution.
        \begin{itemize}
            \item Depending on how important it is to separate the catalyst for your given application, maybe you do the precipitation procedure 4-5 times.
        \end{itemize}
    \end{itemize}
    \item \textbf{Cationation}: This is the general attack of the cation by the monomer.
    \item Benzylic and tertiary halides can also be used to generate ions.
    \begin{itemize}
        \item Benzylic \emph{primary} carbocations are still super stable due to resonance.
    \end{itemize}
    \item General initiation with Lewis acids.
    \begin{itemize}
        \item A two-step process.
        \item If step 2 is the slow step in initiation, then the rate of initiation is $k_i[\ce{M}][\ce{Y+(IZ)-}]$.
        \item Of course, we don't know $[\ce{Y+(IZ)-}]$, but we can work it out from the first equilibrium. This gives us the rate of initiation as
        \begin{equation*}
            R_i = Kk_i[\ce{M}][\ce{I}][\ce{ZY}]
        \end{equation*}
        \item Execise for the student: How do the kinetics change if step 1 is the slow step?
        \begin{itemize}
            \item ...
        \end{itemize}
    \end{itemize}
    \item 2 Propagation.
    \begin{enumerate}[label={\Alph*)}]
        \item Normal propagation: Addition.
        \item Isomerization during propagation: 1,2-hydride or 1,2-methide shifts.
        \begin{itemize}
            \item Driven by wanting to get tertiary carbocations.
            \item Extent of rearrangements depend on two things.
            \begin{enumerate}[label={\arabic*)}]
                \item Relative stabilities of the propagating and rearranged carbocations.
                \item Relative rates of propagation and rearrangement.
            \end{enumerate}
            \item Example: Little or no isomerization occurs with styrene, vinyl ethers, isobutylene because both (1) and (2) favor propagation.
            \item However, this is not the case for some other hydrocarbon-substituted alkenes.
            \begin{itemize}
                \item When you do this at the right temperature, you can get copolymers!
            \end{itemize}
            \item Less isomerization at higher temperatures, which favor propagation.
            \begin{itemize}
                \item At lower temperatures, carbocation stability is king.
            \end{itemize}
            \item Example: $\beta$-pinene can be polymerized.
            \emph{picture}
            \begin{itemize}
                \item Release of ring strain from four-membered ring gives you a new repeat structure.
            \end{itemize}
            \item Another exercise for the student: Figure out how 4,4-dimethyl-1-pentene, ..., polymerize.
            \begin{itemize}
                \item Rowan likes to use these on exams, so don't neglect this suggestion!!
            \end{itemize}
        \end{itemize}
    \end{enumerate}
    \item 3 Chain transfer and termination.
    \begin{itemize}
        \item There are several ways in which chain transfer and/or termination can occur.
        \begin{itemize}
            \item ...
        \end{itemize}
    \end{itemize}
    \item A. 1. Chain transfer to monomer.
    \begin{itemize}
        \item Not much you can do about this.
        \item The monomer would rather attack the positive charge, but it will, eventually, attack one of the protons that has gained a positive charge by stabilizing the carbocation.
        \item NMRs of poly-isobutylene reveal lots of terminal double bonds.
        \item General rate equation:
        \begin{equation*}
            R_{tr,M} = k_{tr,M}[\ce{YM_nM+(IZ)-}][\ce{M}]
        \end{equation*}
        \item The chain-transfer constant to monomer is defined as the relative rates of chain transfer...
        \item The Mayo equation holds true here as well!
    \end{itemize}
    \item A. 2. Hydride ion abstraction from the monomer.
    \begin{itemize}
        \item Transfer the other way around! A proton from the monomer gets picked off by the chain.
    \end{itemize}
    \item B. 1. Chain transfer to the counterion.
    \begin{itemize}
        \item This is a rearrangement of the propagating ion pair.
        \item Kinetically does not depend on the concentration of the monomer.
        \begin{equation*}
            R_{ts} = k_{ts}[\ce{YM_nM+(IZ)-}]
        \end{equation*}
    \end{itemize}
    \item B. 2. Combination with counterion (spontaneous termination).
    \begin{itemize}
        \item One of the main termination processes in TFA-initiated polymerization of styrene.
        \item The counterion just directly bonds to our growing cation on the chain.
        \item TFA $\gg$ \ce{HOAc} because the conjugate base is much more stable.
    \end{itemize}
    \item B. 3. Combination with a fragment of the counterion.
    \begin{itemize}
        \item This is addition of an anionic fragment of the counterion to the polymer.
        \item We cleave the weakest bond in every case.
        \item Example: Since $\ce{B-F}>\ce{B-O}>\ce{B-Cl}$, we append an \ce{OH} preferentially over an \ce{F}, but a \ce{Cl} preferably over an \ce{OH}.
        \item ...
    \end{itemize}
    \item C. 1. Backbiting.
    \begin{itemize}
        \item ...
    \end{itemize}
    \item C. 2. Chain transfer to polymer.
    \begin{itemize}
        \item This is the carbocation of one polymer chain reacting with another polymer chain. Usually a hydride transfer process.
        \item More important at higher conversions for the simple reason that there's more polymer around!
    \end{itemize}
    \item D. Other chain transfer and termination reactions (solvent, added components).
    \begin{itemize}
        \item Everything can kill you, solvent included!
        \item This is why you have to think about your system so carefully as you're designing it.
        \item Rate.
        \begin{equation*}
            R_{tr,S} = k_{tr,S}[\ce{YM_nM+(IZ)-}][\ce{S}]
        \end{equation*}
        \item Deliberately added compounds are good nucleophiles...
        \begin{itemize}
            \item Can be used to functionalize your chain end if you do it right.
        \end{itemize}
    \end{itemize}
    \item Kinetics.
    \begin{itemize}
        \item Assume steady-state conditions exist: At some time in the polymerization, the rate of initiation equals the rate of termination, i.e., every time one dies, another begins.
        \item The total rate of polymerization is
        \begin{equation*}
            R_p = \frac{Kk_ik_p[\ce{I}][\ce{ZY}][\ce{M}]^2}{k_t}
        \end{equation*}
        \begin{itemize}
            \item Important takeaway: $R_p\propto[\ce{M}]^2$.
        \end{itemize}
        \item The number-average degree of polymerization comes down to the Mayo equation again. More specifically, derivation-wise,
        \begin{equation*}
            \overline{X}_n = \frac{R_p}{R_t}
            = \frac{k_p[\ce{YM+(IZ)-}][\ce{M}]}{k_t[\ce{YM+(IZ)-}]}
            = \frac{k_p[\ce{M}]}{k_t}
        \end{equation*}
        \item When chain breaking involves chain transfer to monomer, spontaneous termination, and/or chain transfer to solvent/impurity, then all these processes generated new propagating species.
        \item If we go through the whole derivation, we should get to the Mayo equation
        \begin{equation*}
            \frac{1}{\overline{X}_n} = \frac{k_t}{k_p[\ce{M}]}+\frac{k_{ts}}{k_p[\ce{M}]}+C_M+C_S\frac{[\ce{S}]}{[\ce{M}]}
        \end{equation*}
        \begin{itemize}
            \item Every time we add another chain transfer agent, we can just add another term!
        \end{itemize}
        \item The kinetics if reaction with solvent or chain-transfer agent terminates the polymerization.
        \item $R_p$ becomes
        \begin{equation*}
            R_p = \frac{Kk_ik_p[\ce{I}][\ce{ZY}][\ce{M}]^2}{...+k_t}
        \end{equation*}
        \begin{itemize}
            \item This results in retardation.
        \end{itemize}
    \end{itemize}
    \item Comparison of some rate constants.
    \begin{itemize}
        \item Some numbers for our edification.
    \end{itemize}
    \item General trends for $C_M$ (chain transfer to monomer) and $C_S$ (chain transfer to chain-transfer agent).
    \begin{enumerate}
        \item As $k_p$ increases, $C_M$ decreases:
        \begin{equation*}
            C_K = \frac{k_{tr,M}}{k_p}
        \end{equation*}
        \item $C_S$ is higher for more nucleophilic \ce{S}.
    \end{enumerate}
    \item Effect of reaction medium.
    \begin{itemize}
        \item Solvent and counterion can play a significant role in cationic polymerization.
        \item Example: Diluting \ce{BCl3 + H2O} with a bit of DCM gives us really nice polymerization.
    \end{itemize}
    \item Activation energy of olefin polymerizations.
    \begin{itemize}
        \item Generally, all olefin polymerizations are exothermic reactions on account of the conversion of $\pi$-bonds to $\sigma$-bonds, regardless of the initiator used.
        \item However, the activation energy for the reaction...
        \item All of this follows from the relationship between the rate constant and temperature given by the Arrhenius equation...
    \end{itemize}
    \item Activation energy of olefin polymerizations: Rate.
    \begin{itemize}
        \item $E_p$ does not require a large activation energy.
        \item Anionic olefin polymerization $E_R$ is generally low and positive.
        \item We can get a negative activation energy overall for the reaction if $E_t>E_i+E_p$.
        \begin{itemize}
            \item Colder temperatures give us a faster reaction in this case!
        \end{itemize}
        \item Various polymerizations of styrene, some in which we want to heat it up to make it go faster, and some in which we want to cool it down to make it go faster.
    \end{itemize}
    \item Activation energy of olefin polymerizations: Degree of polymerization.
    \begin{itemize}
        \item We have that $E_{\overline{X}_n}=E_p-E_t$.
        \item Here, lower temperature gives us higher molecular weight.
        \item At $-\SI{100}{\celsius}$, we've effectively switched off chain transfer to solvent.
        \item Degrees of polymerization can increase orders of magnitude by switching off the chain transfer reactions at low temperatures.
    \end{itemize}
\end{itemize}



\section{???}
\begin{itemize}
    \item \marginnote{1/25:}Today.
    \begin{itemize}
        \item Finish up cationic.
        \item Controlled/living free radical.
        \begin{itemize}
            \item Start today, finish up on Tuesday.
            \item Then the rest of Tuesday will be a review of our choice.
        \end{itemize}
    \end{itemize}
    \item Recap of cationic.
    \begin{itemize}
        \item Monomers.
        \begin{itemize}
            \item Nucleophilic.
            \item Lot of conjugation.
            \item Ones that form tertiary carbocations (e.g., isobutylene).
        \end{itemize}
        \item Good at low temperatuers.
        \begin{itemize}
            \item Shuts down side reactions, which have higher activation energies.
        \end{itemize}
        \item You can get rearrangements: Hydride (\ce{H-}) and methide (\ce{CH3-}) shifts.
        \begin{itemize}
            \item Often happens to get you to a tertiary carbocation.
            \item Can be to relieve ring strain.
        \end{itemize}
        \item Nucleophilicity of the counterion is very important in determining rate of polymerization vs. rate of termination.
        \begin{itemize}
            \item Large counteranions tend to be better (less nucleophilic).
        \end{itemize}
        \item Many ways for something to die. A lot more than in the anionic or free radical cases.
    \end{itemize}
    \item We now pick up where we left off on Tuesday.
    \item Molecular weight distribution (dispersity).
    \begin{itemize}
        \item At low conversion, all kinetic parameters ($[\ce{M}]$, $k_t$, $k_p$, $k_{tr}$) are constant and molecular weight is constant.
        \item The probability that a propagation ionic chain will propagate rather than terminate is given by
        \begin{equation*}
            p = \frac{R_p}{R_p+R_t+R_{tr}}
        \end{equation*}
        \item It can be shown that
        \begin{equation*}
            \overline{X_n} = \frac{1}{1-p}
        \end{equation*}
        \begin{itemize}
            \item This is just Carothers' equation.
            \item The derivation is in the book.
        \end{itemize}
        \item We also have
        \begin{equation*}
            \overline{X_w} = \frac{1+p}{1-p}
        \end{equation*}
        \item The dispersity
        \begin{equation*}
            D = \frac{\overline{X_w}}{\overline{X_n}} = 1+p
        \end{equation*}
        has a limit of 2 when $p$ is close to 1.
        \begin{itemize}
            \item This is the most probable or \textbf{Flory}/\textbf{Schultz-Flory distribution}.
        \end{itemize}
        \item On account of the several chain-breaking reactions that can occur, $D$ is to be broad, especially at high conversions.
    \end{itemize}
    \item Differences between radical, cationic, and anionic polymerizations.
    \begin{itemize}
        \item Simplifies learning: Just learn the conserved basics for all of them, and then learn the subtle differences between them.
        \item Ionic polymerizations react dramatically to change the polarity and solvating ability of the solvent (ion pairs, solvent separated ion pairs, and free ions). Radical polymerizations do not.
        \item Ionic polymerizations are more sensitive to impurities and rates of reaction are generally faster.
        \item It is difficult to know sometimes whether a particular initiator system initiates a polymerization by a radical or ionic mechanism: Addition of certain radical scavengers such as the DPPH radical will halt radical polymerizations but not ionic polymerizations. (Note, not true for all radical scavengers.)
    \end{itemize}
    \item Comparing radical and ionic polymerizations.
    \begin{itemize}
        \item The cationic polymerization is described by the equations on the left below, and radical by the right.
        \begin{align*}
            R_p &= \frac{R_ik_p[\ce{M}]}{k_t}&
                R_p &= k_p[\ce{M}]\left( \frac{R_i}{2k_t} \right)^{1/2}\\
            R_p &\propto R_i \propto k_p/k_t&
                R_p &\propto (R_i)^{1/2} \propto k_p/k_t^{1/2}\\
            R_t &\propto [\ce{M+}]&
                R_t &\propto [\ce{M*}]^2
        \end{align*}
        \item Some important differnces.
        \begin{itemize}
            \item Recall that we have the square dependence because the dominant mode of radical termination involves two radicals coming together (e.g., to form a single bond or via disproportionation to form a double bond).
            \item Two positive charges will not come together to terminate!
        \end{itemize}
        \item We do radical reactions at low radical concentration to cut down on termination because termination is so heavily dependent on radical concentration!
        \item Cationic rate of propagation is up to \num{e4} faster than radical rate of propagation.
    \end{itemize}
    \item Designing cationic living polymerizations.
    \begin{itemize}
        \item Living gave us much more control for anionic; can we do something similar for cationic?
        \item Challenge: Redesign so that propagating centers lower reactivity so that transfer and termination reactions are suppressed.
        \item How do we eliminate chain transfer and termination?
        \item Major problem: The distributed positive charge makes $\beta$-protons susceptible to nucleophilic attack (i.e., chain transfer to monomer). In order to prevent this transfer, we will need to stabilize the carbocation by donating electrons to it, which in turn reduces the charge on the $\beta$-protons. But we cannot make the monomer too stable, or it will not polymerize.
    \end{itemize}
    \item Dynamic (reversible) "stabilization" of the reactive carbocation for better control.
    \begin{itemize}
        \item How can we partially break a \ce{R-X} bond without completely breaking it?
        \item It should be noted that this equilibrium (as well as rates of propagation and transfer) is sensitive to a number of factors (e.g., temperature, solvents) and these can determine whether a polymerization is living or not.
        \item Three general methods have been used to achieve living polymerizations.
        \begin{enumerate}
            \item Initiator (\ce{HX})-mild Lewis acid, e.g., \ce{HI/I2}.
            \item Initiator (\ce{HX})-strong Lewis acids + nucleophile, e.g., 1,4-dioxane.
            \begin{itemize}
                \item \emph{see blackboard picture from class}
            \end{itemize}
            \item Initiator (\ce{HX})-strong Lewis acids + salt, e.g., ...
        \end{enumerate}
        \item The initiator may be either an acid (\ce{HX}) or their adducts with vinyl monomers.
    \end{itemize}
    \item Example: Initiator (\ce{HX})-mild Lewis acid.
    \begin{itemize}
        \item The mild Lewis acid (e.g., \ce{ZnX2}) is important.
        \begin{itemize}
            \item If it is too strong, then uncontrolled polymerization occurs.
        \end{itemize}
        \item Vinyl ethers.
        \item Other example.
        \begin{itemize}
            \item It's really important that delocalization makes the carbonyl oxygen \emph{always} the more nucleophilic species in an ester.
            \emph{See picture from blackboard}
        \end{itemize}
    \end{itemize}
    \item Key points for cationic polymerizations.
    \begin{itemize}
        \item Review of the major slides in the presentation.
    \end{itemize}
    \item We now move onto controlled ("living") polymerizations.
    \item Introduction.
    \item \textbf{Living polymer}: A polymer that retains its ability to propagate for a long time and grow to a desired maximum size while their degree of termination of chain transfer is still negligible.
    \item Basic properties.
    \begin{itemize}
        \item ...
    \end{itemize}
    \item If these basic properties are observed, then\dots
    \begin{itemize}
        \item The MWD (dispersity, $D$) should correspond to a Poisson distribution, i.e., 1.
        \item ...
    \end{itemize}
    \item Living anionic polymerizations vs. standard free radical polymerizations.
    \item The difference between living and nonliving polymerizations.
    \begin{itemize}
        \item An ideal free radical polymerization has termination by recombination has a broad distribution.
        \item An ideal living polymerization has a spike in termination.
        \item Anionic polymerization.
        \begin{itemize}
            \item No termination.
            \item No chain transfer.
            \item All chains born at the same time and live until we kill them.
        \end{itemize}
        \item Free radical polymerization.
        \begin{itemize}
            \item Termination present.
            \item Chain transfer present.
            \item All chains are born at different times and we have little control over their death.
        \end{itemize}
        \item Almost all free radical polymerizations can be done living. Then we're set!
    \end{itemize}
    \item Standard radical polymerizations.
    \begin{itemize}
        \item Free radical polymerization mechanism.
        \begin{itemize}
            \item We need to remove coupling and disproportionation.
        \end{itemize}
    \end{itemize}
    \item How to reduce the concentration of the active species.
    \begin{itemize}
        \item Use a reversible termination process.
        \item Have a stable dormant species that can break into stable radicals, react, and then give you more stable dormant species.
        \item We could get some bulk bimolecular termination, but if we keep radical concentration low with this reversible termination, then we should be pretty good!
    \end{itemize}
    \item Reversible-deactivation radical polymerizations.
    \begin{itemize}
        \item NMP, 1985.
        \begin{itemize}
            \item First one.
        \end{itemize}
        \item ATFP.
        \item RAFT.
    \end{itemize}
    \item Stable free radical polymerization (SFRP).
    \begin{itemize}
        \item We do need \ce{Z*} to be a stable radical; if it reacts with itself, it will push the equilibrium toward the active radical via Le Ch\^{a}telier Principle.
        \item Stable radicals include nitroxide, trityl, and triazolinyl.
        \begin{itemize}
            \item Nitroxide is stable because \ce{O-O} bonds are notoriously weak.
            \item Trityl is stable due to resonance delocalization and sterics.
            \item Triazolinyl is sterics, too.
        \end{itemize}
        \item Nitroxide works the best by far.
    \end{itemize}
    \item Nitroxide-mediated polymerization (NMP).
    \begin{itemize}
        \item Heat up an alkoxylamine.
        \item This works well, for instance, in the polymerization of styrene where we alternate back and forth between dormant and active to keep $[\ce{M*}]$ low.
    \end{itemize}
    \item The persistent radical effect in living radical polymerization.
    \begin{itemize}
        \item As a bit of bimolecular combination occurs, Le Ch\^{a}telier's principle pushes the equilibrium back toward the dormant species, reducing further termination!
        \item Essentially,
        \begin{equation*}
            [\ce{M_n}*][\ce{X*}] = \text{constant}
        \end{equation*}
        \begin{itemize}
            \item If $[\ce{M_n*}]$ goes down, then $[\ce{X*}]$ goes up.
        \end{itemize}
        \item ...
    \end{itemize}
    \item Nitroxides.
    \begin{itemize}
        \item Most common one is TEMPO. You can just buy it.
        \begin{itemize}
            \item The one on the far right appears to be the best.
        \end{itemize}
        \item You need fast initiation (low half-life); otherwise, much polymerization will occur before you've consumed all initiator, which is \emph{not} a living characteristic.
    \end{itemize}
    \item Polymerization of acrylates.
    \begin{itemize}
        \item Poor control (broad $D$) with acrylates.
        \item This is on account of the difference in rate of propagation of acrylates and styrene.
        \begin{itemize}
            \item Acrylates: $k_p=\SI{11000}{\liter\per\mole\per\second}$ at \SI{120}{\celsius}.
            \item Styrene: $k_p=\SI{1800}{\liter\per\mole\per\second}$ at \SI{120}{\celsius}.
        \end{itemize}
        \item Need to slow down the reaction in order to afford better control.
        \begin{itemize}
            \item This can be done by the addition of free nitroxide.
        \end{itemize}
        \item Takeaway: There are games to play with these kinds of reactions; if you think about it the right, you can basically work out what you need to do.
    \end{itemize}
    \item Atom transfer radical polymerization (ATRP).
    \begin{itemize}
        \item Pioneered by Sawamoto and Namacheski.
        \item The metal pulls off a halide to create a radical that can polymerize, then can react back to a dormant species.
        \item Very similar to NMP.
        \item Requires a redox-active transition metal that can do 1-electron chemistry, e.g., \ce{Cu^I/Cu^{II}} (most common today), \ce{Ru^{II}/Ru^{III}}, \ce{Fe^{II}/Fe^{III}}.
        \item The catalyst needs to have\dots
        \begin{enumerate}
            \item Two oxidation states easily accessed by a 1-electron transfer;
            \item An affinity for halogens/pseudohalogens \ce{X};
            \item ...
        \end{enumerate}
    \end{itemize}
    \item Copper-catalyzed ATRP.
    \begin{itemize}
        \item Use \ce{Cu^{I}Br(L)}.
        \begin{itemize}
            \item \ce{Cu^{II}Br2(L)} is our "persistent (metallo)radical."
        \end{itemize}
        \item The equilibrium constant $K$ is important here ($\sim\num{e-6}$) and provides an appropriate measure of the catalyst's activity in a polymerization reaction.
        \begin{itemize}
            \item If $K$ is too small, then the reaction will not proceed.
            \item If $K$ is too large, then a large amount of termination will occur (too many radicals).
        \end{itemize}
    \end{itemize}
    \item Examples of monomers, initiators, and ligands used in ATRP.
    \begin{itemize}
        \item Monomers.
        \begin{itemize}
            \item Extensive.
            \item Styrenes, acrylates, etc.
        \end{itemize}
        \item Initiators.
        \begin{itemize}
            \item A lot of things with reactive species at the end.
        \end{itemize}
        \item Ligands.
        \begin{itemize}
            \item Bidentate, tridentate, tetradentate, etc.
            \item You just tune this until stuff works!
        \end{itemize}
    \end{itemize}
    \item If you do not select the correct initiator, ligand, and contitions for ATRP for a given monomer, the reaction will not be controlled.
    \begin{itemize}
        \item There's a crazy amount of work out there on ATRP, so start by looking for a paper that's already done something similar and go from there.
        \item Don't use the first paper you find!
    \end{itemize}
    \item Copper-catalyzed ATRP: Kinetics.
    \begin{itemize}
        \item You only add 2-3 units at a time if you have a fast equilibrium.
        \item You may also add hundereds of units each time with a slow equilibrium.
        \begin{itemize}
            \item In this case, subsequent polymerizations will have noticably lower monomer concentrations, so you get higher variation in molecular weights.
        \end{itemize}
        \item Assuming no termination and a fast equilibrium,
        \begin{align*}
            [\ce{RM*}] &= \frac{K[\ce{I}][Cu^+]}{[\ce{Cu^2+}]}&
            [R_p] &= ...
        \end{align*}
        \item Really nice straight plots...
    \end{itemize}
    \item Copper-catalyzed ATRP ($D$).
    \begin{itemize}
        \item We have
        \begin{equation*}
            D = \frac{\overline{X_w}}{\overline{X_n}}
            = 1+\left( \frac{[\ce{I}]_0k_p}{k_d[\ce{Cu^2+}]} \right)\left( \frac{2}{p}-1 \right)
        \end{equation*}
        \item Lower $D$: Low initiator concentration, higher conversions, rapid deactivation (high $k_d$ and $[\ce{Cu^II}]$) and lower $k_p$.
        \item Proposed \ce{Cu^I/Cu^{II}} species using bpy as a ligand.
        \item The equilibrium is important for both rate and dispersity.
        \begin{itemize}
            \item $K$ increases with solvent dispersity and temperature.
            \item ...
        \end{itemize}
    \end{itemize}
    \item Copper-catalyzed ATRP: Effect of ligand.
    \emph{picture}
    \begin{itemize}
        \item We're not expected to remember this, but this is a very useful slide.
        \item "If you're doing an ATRP, you're welcome" - Rowan.
        \item General trends: Copper complex activity increase (larger $k_a=k_\text{act}$).
        \begin{itemize}
            \item $\text{tetradentate}>\text{terdentate}>\text{bidentate}$.
            \item $\text{pyridine}>\text{aliphatic amine}>\text{imine}$.
            \item Sterics; \ce{Me6TREN} is 10,000 times more active than \ce{Et6TREN}.
        \end{itemize}
    \end{itemize}
\end{itemize}




\end{document}