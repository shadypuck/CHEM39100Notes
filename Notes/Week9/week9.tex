\documentclass[../notes.tex]{subfiles}

\pagestyle{main}
\renewcommand{\chaptermark}[1]{\markboth{\chaptername\ \thechapter\ (#1)}{}}
\setcounter{chapter}{8}

\begin{document}




\chapter{???}
\section{???}
\begin{itemize}
    \item \marginnote{2/27:}Bisphenol A Polycarbonate.
    \begin{itemize}
        \item Good optical properties (clarity).
        \item When you microwave baby bottles with it, though, you get leaching.
    \end{itemize}
    \item Polycarbonate synthesis.
    \begin{itemize}
        \item Interfacial polycondensation.
        \begin{itemize}
            \item Rapidly stirred mixture of solvent (e.g., DCM, PhCl, DCE), aqueous NaOH, bisphenol A, phosgene, a tertiary amine catalyst, and a monofunctional end-capper for MW control.
            \item IN the aqueous phase, there are some reactions. In the organic, there are others. At the interface, there are still more.
        \end{itemize}
        \item You can also do transesterification.
        \begin{itemize}
            \item Melt of Bisphenol A, diphenylcarbonate, sodium bisphenolate catalyst.
            \item High temperature and low pressure.
            \item You get some undesired side reactions.
        \end{itemize}
    \end{itemize}
    \item Polyamides (nylons).
    \begin{itemize}
        \item Linear aliphatic polyamides will be our focus.
        \item The first one was "Polyamide x," aka PerlonTM.
        \item "Polyamide x,y" was NylonTM.
    \end{itemize}
    \item Polyamides History.
    \begin{itemize}
        \item Kirby and Carothers (1930) failed to polymerize caprolactam, and put this result in writing in JACS stating that it couldn't be done. German IG Farben was then able to do it and patent it.
        \item ...
        \item Peterson in Carothers' lab made polyamide 5-10 with even better characteristics.
        \item Triggered by the business people at DuPont, who were concerned that polyamide 5-10 was too expensive to make, Carothers' group started to screen...
        \item 1935: Berchet prepared the first sample of polyamide 6,6. The polymer had exciting properties (although the high melt temperature suggested possible problems due to degradation upon processing) and the economics of the monomers were right, so that in summer 1935, polyamide 6,6 was selected as the commercial candidate.
        \begin{itemize}
            \item The "solidification" was a cold crystallization.
        \end{itemize}
        \item 1939: Polymer brought to market.
    \end{itemize}
    \item Polyamide.
    \begin{itemize}
        \item List of properties.
        \begin{itemize}
            \item Moderate to high crystallinity.
            \item High water uptake (wet nylon is softer).
            \begin{itemize}
                \item The keratin in our nails is polyamide, too, with some side groups. Thus, it has high water uptake, too. That's why our nails get soft in the bath!
            \end{itemize}
            \item Few solvents: Hexafluoro-2-propanol, formic acid, \emph{m}-cresol.
        \end{itemize}
        \item ...
    \end{itemize}
    \item Application of polyamides.
    \item Polyamide fibers.
    \begin{itemize}
        \item We get a lot of hydrogen bonding that stops chains from slipping past each other.
        \item Hydrogen bonding is stronger than the dipole interactions in polyesters (use in gen-chem review??), creating ordered crystalline polymer chains.
    \end{itemize}
    \item Polyamide 6,6 synthesis.
    \begin{itemize}
        \item The nylon rope trick, forming nylon via interfacial polymerizatio.
        \begin{itemize}
            \item If the rate of reaction is faster than the rate of pulling up the polymer, you'll get a nice long rope.
        \end{itemize}
        \item Melt polycondensation.
        \begin{itemize}
            \item Steps to do it.
        \end{itemize}
    \end{itemize}
    \item Polyurethanes.
    \begin{itemize}
        \item Very easy to make.
        \item You can do the reverse reaction with catalysts above \SI{180}{\celsius}.
        \item Most polymerization is done with a catalyst under \SI{100}{\celsius}, though.
        \item These can be made as either fibers or foams.
        \item Lists a number of possible di(isocyanates) and diols.
        \item There are literally thousands of polyurethanes.
        \begin{itemize}
            \item Some are very stiff, e.g., polyurethanes are used in bowling balls.
            \item Some are very soft and stretchy, e.g., spandex and shower curtains.
        \end{itemize}
        \item Polyester leisure suits in the '70s, spandex in the '80s. As a polymer chemist, with great power comes great (fashion) responsibility.
    \end{itemize}
    \item Polyurethane fibers.
    \begin{itemize}
        \item Commercially, the most important polyurethane fibers are those that result in a blocky copolymer structure (e.g., spandex).
        \item To make these, a low molecular weight polymeric diol ($M_n=\numrange{2000}{3000}$) is reacted with excess isocyanate to result in isocyanate-terminated polymer which is then reacted with a diamine chain extender.
    \end{itemize}
    \item Polyurethane foams.
    \begin{itemize}
        \item ...
    \end{itemize}
    \item Aromatic polymers.
    \begin{itemize}
        \item Polysulfone (PSU) is a rigid, high-strength, semi-tough thermoplastic that has a \textbf{heat deflection temperature} of \SI{174}{\celsius} and maintains its properties over a wide temperature range.
        \begin{itemize}
            \item Synthesized via a nucleophilic substitution-type mechanism.
        \end{itemize}
        \item Polyetheretherketone (PEEK).
        \item EWGs to pull electrons up before they kick back down are critical in the synthesis for both of these.
        \item Poly(imide).
        \begin{itemize}
            \item Really good strong materials.
            \item Nucleophilic aromatic substitution mechanism here, too.
            \item You go through a \textbf{polyamic acid} intermediate. This is solution processible. Then when you "heat the bajesus out of it," you form the polyimide.
        \end{itemize}
        \item Poly(ether imide).
        \begin{itemize}
            \item Same NAS mechanism.
        \end{itemize}
        \item Poly(phenylene oxide).
        \item Poly(phenylene sulfide).
    \end{itemize}
    \item \textbf{Heat deflection temperature}: The temperature at which a polymer starts to "soften" under a fixed load.
    \item Reminder: Most polymer chemistry is done under extreme conditions that we wouldn't ordinarily find in the lab.
    \begin{itemize}
        \item We often try to use more mild conditions when we have more functional groups, as in polyurethanes, though.
    \end{itemize}
    \item Branched and network polymers.
    \begin{itemize}
        \item We'll start this today, finish it on Thursday, and then half the class will be left. Rowan is open to any activity we want to propose. He can go over anything again, etc.
    \end{itemize}
    \item Reactions with monomers with nore than two functional groups.
    \begin{itemize}
        \item All the step-growth polymerizations so far have focused on monomers with only two reactive groups which lead to the formation of either linear polymers or rings.
        \item Consider the polymerization of \ce{A-B} monomer in the presence of a small amount of \ce{A}... \textbf{functionality}
    \end{itemize}
    \item Crosslinking.
    \begin{itemize}
        \item Reacting \ce{A-B} with \ce{A_f} in the presence of \ce{B-B} leads to branching and crosslinking.
        \item Crosslinking also occurs with
        \begin{align*}
            \ce{A-B + B_f} &\ce{->}&
            \ce{A-A + B-B + B_f} &\ce{->}&
            \ce{A_f + B_f} &\ce{->}
        \end{align*}
        \item Crosslinking is distinguished by the occurrence of \textbf{gelation} during the polymerization.
        \begin{itemize}
            \item At this gel point, an insoluble polymer fraction (\textbf{gel}) is formed.
            \item It's all just one big ass molecule.
        \end{itemize}
    \end{itemize}
    \item Networks: Definitions.
    \item \textbf{Junction}: Point where three or more strands emanate.
    \item \textbf{Functionality} (of a junction): The number of strands connected to it. \emph{Denoted by} $\bm{f}$.
    \item \textbf{Strand}: A polymer chain that bridges two junctions.
    \item \textbf{Loop}: A network defect that begins and ends at the same junction point and is not fully elastically active.
    \item \textbf{Dangling chain end}: A network defect that is free to relax and as such does not contribute to the network elasticity.
    \item Loops and dangling chain ends are \emph{defects}.
    \item \textbf{Sol fraction}: This is the fraction of polymer that is not part of the invite network either as a linear polymer or as part of a cluster.
    \begin{itemize}
        \item If put in a good solvent, the sol fraction would dissolve and the \textbf{gel fraction} would not.
        \item After you synthesize your polymer, you identify the gel fraction by washing it, drying it, and weighing it.
    \end{itemize}
    \item \textbf{Networks}: A class of polymers that consists of an infiinte crosslinked architecture.
    \begin{itemize}
        \item The linkages prevent flow of the material, and as such, these materials are solids.
        \item Two main classes of networks: \textbf{elastomers} and \textbf{thermosets}.
    \end{itemize}
    \item \textbf{Elastomer}: ...
    \item \textbf{Thermoset}: ...
    \item \textbf{Gel point}: The point during the polymerization at which gelation occurs.
    \begin{itemize}
        \item A key property of crosslinked polymers is that they never dissolve. They may swell, but they will never dissolve unless we break bonds.
        \item Definition of the \textbf{sol}.
        \item Once the gel point is reached, the polymer will no longer flow (even at high temperatures). Thus, processing needs to be completed before gelation sets in.
        \item Thus, the \textbf{prepolymer} is processed before gelation is reached, and polymerization/crosslinking is completed after processing.
        \item Three stages for thermosetting polymers or thermosets.
        \begin{enumerate}
            \item Polymer soluble and fusible.
            \item Polymer still fusible but close to the gel point.
            \item Polymer highly crosslinked...
        \end{enumerate}
    \end{itemize}
    \item \textbf{Sol}: Any non-gel portion of the polymer which remains soluble.
    \item Network synthesis.
    \begin{itemize}
        \item It is important to understand the relationship between extent of reaction and gelation. This can be done based on calculating when either $\overline{X}_n$ and $\overline{X}_w$ reach an infinite size.
        \item Carothers equation: $\overline{X}_n\to\infty$.
        \item You have to think about the stoichiometric amount of reagents.
        \item This is based on the average functionality of the monomer mixture, i.e., the average number of functional groups per monomer molecule:
        \begin{equation*}
            f_\text{avg} = \frac{\sum N_if_i}{\sum N_i}
        \end{equation*}
        \begin{itemize}
            \item Recall that $N_i$ is the number of molecules of monomer $i$ with functionality $f_i$.
            \item Example: For 2 mol of glycerol (triol) and 3 mol of phthalic acid (diacid), then
            \begin{equation*}
                f_\text{avg} = \frac{12}{5} = 2.4
            \end{equation*}
        \end{itemize}
        \item For the system where the number of \ce{A} and \ce{B} groups are equal, the number of monomers...
        \item ...
        \item Combining equations, we get
        \begin{equation*}
            \overline{X}_n = \frac{2}{2-pf_\text{avg}}
        \end{equation*}
        \item Rearranging, we obtain
        \begin{equation*}
            p = \frac{2}{f_\text{avg}}-\frac{2}{\overline{X}_nf_\text{avg}}
        \end{equation*}
        \begin{itemize}
            \item This relates the extent of reaction and degree of polymerization to the average functionality in the system.
        \end{itemize}
        \item The gel point is where $\overline{X}_n=\infty$. Thus, the \textbf{critical extent of the reaction} $p_c$ when gelation occurs is
        \begin{equation*}
            p_c = \frac{2}{f_\text{avg}}
        \end{equation*}
        \item ...
        \item When the two functional groups are not at stoichiometry, then the average functionality of themolecules is twice that of the amount of the functional group not in excess divided by the numbeff of molecules, i.e., the extent of crosslinking (if it occurs) depends on the deficient functionality.
        \item Example:
        \emph{picture of glycerol and phthalic acid.}
        \begin{itemize}
            \item ...
        \end{itemize}
        \item Extension to nonstoichiometric reactant mixtures.
        \item The average functionality of a system containing more than two monomers can be calculated in a similar manner.
        \item Consider a three component system...
    \end{itemize}
    \item Experimental gel points.
    \begin{itemize}
        \item The Carothers (and statistical theoretical approaches not covered here) differ in their predictions of extent of reaction, $p_c$.
        \item The gel point is usually determined experimentally as the point in the reaction at which the mixture loses fluidity (as indicated by the failure of bubbles to rise).
        \item The data shows that gelation usually happens a bit before the Carothers prediction and afte the statistical prediction.
        \item ...
    \end{itemize}
\end{itemize}



\section{???}
\begin{itemize}
    \item \marginnote{2/29:}I was at KHS Immersion Weekend.
\end{itemize}




\end{document}