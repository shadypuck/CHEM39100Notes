\documentclass[../notes.tex]{subfiles}

\pagestyle{main}
\renewcommand{\chaptermark}[1]{\markboth{\chaptername\ \thechapter\ (#1)}{}}
\setcounter{chapter}{4}

\begin{document}




\chapter{???}
\section{???}
\begin{itemize}
    \item \marginnote{1/30:}Review of copper-catalyzed ATRP ligand effects.
    \begin{itemize}
        \item They're trying to prevent termination by lowering the radical concentration.
        \item Essentially, the ligand plays an important role in the rate of activation.
    \end{itemize}
    \item Copper-catalyzed ATRP: Effect of initiator.
    \begin{itemize}
        \item Initiator activity increase is related to the stabilization of the generated radical, so $3^\circ>2^\circ>1^\circ$...
    \end{itemize}
    \item Copper-catalyzed ATRP: ppm Cu.
    \begin{itemize}
        \item Residual copper could have adverse effects if you're looking to make a medical product that will go into a person or a polymer to go into a piece of technology.
        \item We typically need 0.1-1 mol\% of copper relative to the monomer, so the product will contain a lot of it.
        \item Thus, if we want to filter it out beyond just precipitating our polymer, we can throw in chelating agents or do fancier things, but that all adds cost and time to the process.
        \item Termination does occur.
        \item The rate of reaction depends on the ratio of $[\ce{Cu^+}]/[\ce{Cu^2+}]$ and not the absolute catalyst concentration.
        \begin{itemize}
            \item So theoretically, we can go to much smaller catalyst loadings, as long as we keep that ratio high.
        \end{itemize}
        \item However, this is difficult to do because we need to regenerate \ce{Cu^+} using something other than radicals.
    \end{itemize}
    \item Copper-catalyzed AGET and ARGET ATRP.
    \begin{itemize}
        \item AGET (Activators Generated by Electron Transfer) ATRP: Stoichiometric amounts of reducing agent added to the reaction mixture containing alkyl halides...
        \begin{itemize}
            \item Wrapping copper wire around your stir bar will usually do the trick.
            \item Essentially, we're just adding in a reducing agent.
        \end{itemize}
        \item ARGET (Activators ReGenerated by Electron Transfer) ATRP: Similar to AGET, but we use an excess of reducing agent which allows the use of much lower amounts of copper.
        \item We can also use electrochemistry of photochemistry to reduce down.
        \item These methods are much more commonplace now than traditional ATRP, \emph{especially} if we're working toward biomedical devices.
    \end{itemize}
    \item Reversible Addition-Fragmentation Chain Transfer (RAFT).
    \begin{itemize}
        \item You use a standard free radical termination, but instead of having reversible termination, you have a reversible chain transfer process.
    \end{itemize}
    \item Discovery of RAFT polymerization (1998).
    \begin{itemize}
        \item RAFT polymerization controls chain growth through reversible chain transfer.
        \item It can be achieved by the introduction of a small percent of a chain transfer agent into a conventional free-radical polymerization.
        \item We most commonly use dithioester derivatives.
        \item The \ce{R} group...
    \end{itemize}
    \item Steps in RAFT polymerization.
    \begin{itemize}
        \item We use a standard initiator (e.g., AIBN).
        \item We then have a step of pre-equilibrium or initialization.
        \begin{itemize}
            \item Here, a polymer can add into the dithioester and kick out \ce{R*} to begin a new polymerization.
        \end{itemize}
        \item Re-initialization: The growing of the new polymer from \ce{R*}.
        \item Main equilibrium: Chain transfer back and forth through dormant chains bonded to dithioesters.
        \item Various \ce{R} groups for the chain transfer agent.
    \end{itemize}
    \item RAFT: Snap shot of polymer chains.
    \begin{itemize}
        \item A few dead chains, mostly dormant chains, and a few propagating chains.
        \item Allows you to get a nice spike in molecular weight. Note that you might get a little tail on the low end due to dead chains.
        \item All you have to do in the lab is throw in a bit of your RAFT pixie dust.
    \end{itemize}
    \item RAFT equilibria.
    \begin{itemize}
        \item The efficiency of a RAFT agent can be defined by $C_{tr}$. More active RAFT agents have larger $C_{tr}$. ($C_{tr}$ should be at least 10; many are $>100$.)
        \item We have that $C_{tr}=k_{tr}/k_p$ where
        \begin{equation*}
            k_{tr} = k_{add}\phi
            = k_{add}\frac{k_\beta}{k_{-add}+...}
        \end{equation*}
    \end{itemize}
    \item Kinetics and livingness.
    \begin{itemize}
        \item We won't go in depth into how to derive this stuff.
        \item The rate of polymerization is...
        \item We can define the \textbf{livingness} (how much is dormant or growing vs. dead).
        \begin{equation*}
            L = \frac{[CTA]_0}{[CTA]_0+2f[\ce{I}]_0(1-\e[-k_dt])(1-f_c/2)}
        \end{equation*}
        \item The number of chains produced in a radical-radical termination event.
        \item $f_c=1$ means 100\% termination by combination, whereas $f_c=0$ means 100\% termination by disproportionation.
        \item Monomers with high propagatino rates (such as acrylamides) and initaotors with high efficiency or high decomositio rate lead to fast polymerization rate while keeping low initiator concentration.
        \item An optimal RAFT system requires a high rate of radical generation (considering $k_d$, for instance, by using thermal initiator at $T>$ their ten-hour half-life) and/or solvent induced acceleration.
        \item A large polymerization rate permits shorter polymerization time or lower amounts...
    \end{itemize}
    \item Reactivity of the monomers and their propagating radicals.
    \begin{itemize}
        \item Monomers can be classified into two general groups.
        \item More "activated" monomers (MAMs).
        \begin{itemize}
            \item Conjugated double bond (e.g., styrene, methyl methacrylate, acrylamide, acrylonitrile, \dots).
            \item Reactive monomer.
            \item Less reactive propagating chains.
            \item This means that poly(MAM) is a relatively good homolytic leaving groups, and as such, the more active RAFT agents provide good control. Less active RAFT agents have lower $C_{tr}$ and provide poor control.
        \end{itemize}
        \item Less "activated" monomers (LAMs).
        \begin{itemize}
            \item Saturated carbon or oxygen/nitrogen.
            \item ...
        \end{itemize}
    \end{itemize}
    \item Relationship of RAFT agent and monomer.
    \begin{itemize}
        \item Optimal control in RAFT polymerization requires choosing an appropriate RAFT agent for the monomer(s) to be polymerized.
    \end{itemize}
    \item Z group impacts the reactivity of the RAFT agents.
    \begin{itemize}
        \item How reactive is \ce{C=S} to attack?
        \item The reactivity can be qualitatively understood in terms of the importance of zwitterionic canonical forms. The lone pair delocalization with the \ce{C=S} reduces its double bond character and stabilizes the RAFT agent relative to the radical intermediate.
        \item The reactivity of the RAFT agent can be increased if the electron density of the lone pair is reduced by delocalization everywhere.
    \end{itemize}
    \item Relationship of RAFT agent and monomer; the role of the Z group.
    \begin{itemize}
        \item Z modifies the rate of addition of propagating radicals ($k_{add}$); it can be altered by 5 orders of magnitude.
        \item Z alters the stability of the radical intermediate; the rate of fragmentation $k_{-add}$.
        \item Z should not cause any side reactions.
        \begin{itemize}
            \item For xanthides...
            \item If Z is strongly electron withdrawing, then the \ce{C=S} an react directly with the monomer (e.g., a hetero Diels-Alder) or nucleophilic attack.
        \end{itemize}
        \item Guidelines for selection of the Z group of RAFT agents (\ce{ZC(=S)SR}) for various polymerizations.
        \begin{itemize}
            \item ...
        \end{itemize}
    \end{itemize}
    \item Relationship of RAFT agent and monomer; the role of the R group.
    \emph{picture}
    \begin{itemize}
        \item For an effective RAFT agent, \ce{R*} should be\dots
        \begin{itemize}
            \item A really good leaving group with respect to the propagating radical $[\ce{P*}]$ ($\phi>0.5$);
            \item Be able to reinitiate the polymerization efficiently ($k_{iR}>k_p$), otherwise retardation occurs. Related to the stability of \ce{R*}.
        \end{itemize}
        \item Guidelines for selection of the R group of RAFT agents.
        \begin{itemize}
            \item See Figure ??
            \item A dashed line indicates only partial control (control of molar mass but poor control over dispersity or substantial retardation in the case of VAc, NVC, NVP).
        \end{itemize}
        \item Radical stability is important in determining the fragmentation rates.
        \item Steric factors play a role.
        \item Polymerization of 1,1-disubstituted monomers (e.g., methyl methacrylate, methacrylamides) which result in a tertiary \ce{P_n*}...
        \item However, polymerization of monomers with high $k_p$ are best controlled with primary and secondary R groups. Tertiary radicals are inefficient at reinstating polymerizations since $k_{i,R}$ is often lower than $k_p$.
    \end{itemize}
    \item Just because someone gets it to work for styrene doesn't mean it'll work for everything.
    \item RAFT CTA choice is everything.
    \begin{itemize}
        \item A table that summarizes choices.
    \end{itemize}
    \item Synthetic potential of RAFT.
    \begin{itemize}
        \item You can do chemistry on your RAFT chain ends.
        \item Example: You can react a diene with the \ce{C=S} bond in a hetero Diels-Alder reaction.
    \end{itemize}
    \item Targeting molecular weights with RAFT.
    \begin{itemize}
        \item Your number average molecular weight is
        \begin{equation*}
            \frac{\text{Total monomer concentration}}{\text{Total initiated-unit concentration}}\times\text{Conversion}\times\text{Monomer molar mass}+\text{CTA molar mass}
        \end{equation*}
        \item Thus,
        ...
    \end{itemize}
    \item Key points.
    \begin{itemize}
        \item Understand what a living polymerization is.
        \item What does that mean for the resulting polymerization?
        \item Standard living (ionic) and controlled living (reversible step).
        \item Know and understand RAFT, ATRP, and NMP; know how to decide which one you will use.
    \end{itemize}
    \item Midterm.
    \begin{itemize}
        \item Know common monomers, know common polymers, know how they can be initiated.
        \item Know how to draw reaction mechanisms.
        \item There's at least one question that might require a calculator.
        \begin{itemize}
            \item Implication: There's a kinetics question coming.
        \end{itemize}
        \item For the grad level course, papers are 20\% of the exam! Don't ignore them!!
        \item The kinetics from today's lecture won't be there.
        \item Focus on intro, cationic, anionic, and free radical classes.
    \end{itemize}
\end{itemize}



\section{???}
\begin{itemize}
    \item \marginnote{2/1:}I was at ??
\end{itemize}




\end{document}