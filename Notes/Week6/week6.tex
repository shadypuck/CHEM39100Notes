\documentclass[../notes.tex]{subfiles}

\pagestyle{main}
\renewcommand{\chaptermark}[1]{\markboth{\chaptername\ \thechapter\ (#1)}{}}
\setcounter{chapter}{5}

\begin{document}




\chapter{???}
\section{Ring-Opening Polymerizations}
\begin{itemize}
    \item \marginnote{2/6:}Relevant reading: Odian, Chapter 7.
    \item \textbf{Ring-opening polymerization}: The conversion of cyclic monomers into polymers via a ring opening process. \emph{Also known as} \textbf{ROP}.
    \item Introduction.
    \begin{itemize}
        \item The impact of size is very important in ROP.
        \item You need a way to \emph{kinetically} open the ring.
        \item There can be an \emph{equilibrium} between macrocyclic and linear species.
        \begin{itemize}
            \item This equilibrium is big.
            \item In order to induce polymerization, the equilibrium has to be driven toward the linear side.
        \end{itemize}
        \item Not every ring can be polymerized; one of the big things in this class is learning how to determine which rings can be polymerized.
        \item A monomer require ring strain (thermodynamic) and a nucleophilic site (kinetic).
    \end{itemize}
    \item Monomers that can be polymerized via ROP.
    \begin{itemize}
        \item We'll talk about most of the monomers on this slide.
        \item First half: Polymerizing ethers and ester.
        \begin{itemize}
            \item Then different ones.
        \end{itemize}
        \item Poly-$\varepsilon$-caprolactam is nylon!
        \item There is a polymer called polysulfur that is just a bunch of sulfur atoms in a chain!
    \end{itemize}
    \item General mechanism (two basic types).
    \begin{enumerate}
        \item Cationic ROP.
        \begin{itemize}
            \item We get a positive charge (e.g., oxonium) and then the charge transfers after nucleophilic attack.
        \end{itemize}
        \item Anionic ROP.
        \begin{itemize}
            \item The anion on the growing chain attacks the monomer.
        \end{itemize}
    \end{enumerate}
    \item Major driving force: Ring strain.
    \begin{itemize}
        \item This is a more thermodynamic factor and is the relative stability of the cyclic monomer vs. linear polymer.
        \begin{itemize}
            \item We \emph{can} use entropy to drive this, but most often, we take $\Delta H\ll 0$.
        \end{itemize}
        \item ...
    \end{itemize}
    \item Bond angle distortion.
    \begin{itemize}
        \item This is the primary way to drive polymerization.
        \item 3- and 4-membered rings have the most distorted bond angles and are very reactive toward polymerization.
        \item \textbf{Transannular strain} increases strain in an 8-membered ring.
        \begin{itemize}
            \item 8-membered rings are actually \emph{great} to polymerize.
        \end{itemize}
        \item A 6-membered ring is the most thermodynamically stable; except for a \emph{few} known exceptions, these do not polymerize.
        \begin{itemize}
            \item Example: Trioxane; the oxygens get us away from ideal bond angles because of their lone pairs!
        \end{itemize}
        \item Thus, \emph{en toto}, the thermodynamic feasibility for polymerization in terms of ring number is
        \begin{equation*}
            3,4 > 8 > 5,7 \gg 6
        \end{equation*}
    \end{itemize}
    \item Kinetic factors.
    \begin{itemize}
        \item Require functionality (a nucleophile) in the ring, which allows propagation to occur.
        \item The rate of polymerization depends on both the reactivity of the monomer and active species.
        \begin{itemize}
            \item The question is, "which is more important?" The \emph{nucleophilic} nature of the monomer and the \emph{electrophilic} nature of the active species.
        \end{itemize}
        \item Aziridine is more nucleophilic than cyclopropoxide because the nitrogen is less electronegative than the oxygen.
        \item In general, the higher the basicity of the parent monomer, the lower the reactivity of the active species derived from this monomer toward the standard monomer.
        \item The rate constant $k_p$ of homopropagation overall decreases as the nucleophilicity of Z in the monomer increases because the activity of the active species controls the rate of homopropagation.
        \item This tells us that in the transition state, the bond breaking is more advanced (it's driving the polymerization) than the bond making.
    \end{itemize}
    \item Thermodynamics of polymerization of cyclic ethers and acetals.
    \begin{itemize}
        \item Observe that 3- and 4-membered cyclics are comparable to those for alkenes.
        \item Substitution on the ring generally decreases tendency to polymerize because of sterics.
        \item 6-membered rings are generally difficult to polymerize.
        \item Generally, the larger the ring, the lower the entropic cost.
    \end{itemize}
    \item Comparison to olefin chain and step-growth polymerization.
    \begin{itemize}
        \item Initiators required, similar to cationic and anionic initiators used in olefin polymerizations.
        \item Shows similar characteristics to those described for ionic olefin polymerizations: Effect of solvent, counterion, propagation by different species (covalent, ion pair, free ions).
        \item Growth process similar to chain polymerization in that only monomer adds to growing chain.
        \item However, rate constants are closer to step-growth polymerizations and there is a slower build up of molecular weight...
        \begin{itemize}
            \item This means that it's easier to get living polymerizations with this technique because the reactions are less fast.
        \end{itemize}
    \end{itemize}
    \item The following kinetics are almost identical to the alkene case.
    \item Thus, Rowan will focus more on the \emph{chemistry} than the \emph{kinetics}.
    \item Chemistry of ROPs.
    \begin{itemize}
        \item Very similar to step-growth polymerizations, but with no condensation product.
        \item ...
        \item In THF, the positive charge converts oxygen into a good leaving group and we just do S\textsubscript{N}2.
        \item Aside: You should never use a strong cation in THF solvent because the cation will just polymerize it!
    \end{itemize}
    \item Cationic Ring-Opening Polymerization (CROP).
    \begin{itemize}
        \item Propagation with a \emph{cationic} species.
        \item General characteristics.
        \begin{itemize}
            \item The cyclic monomers require a nucleophilic site. In general, the nucleophilic site in ROP is more nucleophilic than a double bond.
            \item Shows most of the characteristics of vinyl cationic polymerization.
            \item ...
        \end{itemize}
    \end{itemize}
    \item Initiation.
    \begin{itemize}
        \item Similar to cationic!
        \item Use a Br\o nsted (protic) acid.
        \item Stable organic salts work well.
        \item Lewis acids can also help out.
        \item You can also do initiation with a covalent compound.
        \begin{itemize}
            \item Use strong alkylating or acylating agents.
            \item Depends on the nature of the nucleophile in the ring.
            \item \ce{RBr} is good for amines and oxazolines.
            \item \ce{ROSO2CF3} is good for acetals or ethers (O); it's a \emph{very} good \textbf{methylation agent}.
            \begin{itemize}
                \item This also means that you should not use an alkylating agent in THF!
                \item We can think of \ce{CF3SO3} as an almost \emph{dormant} species, because once it is bound to a chain end, it can just react with THF again!
            \end{itemize}
        \end{itemize}
    \end{itemize}
    \item Propagation.
    \begin{itemize}
        \item Three possible propagation mechanisms.
        \begin{enumerate}
            \item Activated chain-end (ACE) mechanism (ionic species).
            \item Activated chain-end (ACE) mechanism (covalent species, pseudocationic).
            \item Activated monomer (AM) mechanism.
        \end{enumerate}
        \item Which propagation mechanism occurs depends on the nucleophile in the ring and the initiator.
        \item Examples of ACE and AM.
        \item Both ACE and AM can occur simultaneously in the polymerization process. The AM mechanism (which reduces the amount of cyclics formed during polymerization) can predominate if the ratio of $[\text{Monomer}]/[\ce{HO-}]$ is kept low, i.e., slow addition of monomer to the reaction mixture.
        \item We can get backbiting in AM mechanisms; polymerizing ethylene oxide cationically just gets you a bunch of dioxane, which is just a thermodynamic well that you fall into.
        \begin{itemize}
            \item Adding a small smidge of alcohol in will preferentially react with the activated monomer.
            \item Not the best ever; ACE will still be actively generating dioxane. But not bad.
        \end{itemize}
    \end{itemize}
    \item Transfer and termination.
    \begin{itemize}
        \item Almost all analogous to cationic alkene polymerization.
        \item Chain transfer to polymer.
        \begin{itemize}
            \item Probably the most common mode of transfer in this case.
            \item If intermolecular, we'll get some mixing of molecular weights as chains exchange their growing fragments.
            \item We can get intramolecular backbiting, too.
        \end{itemize}
        \item Termination by irreversible recombination.
        \begin{itemize}
            \item Three different nucleophiles can react with the active cationic species.
            \begin{enumerate}
                \item Monomer (leads to propagation).
                \item ...
            \end{enumerate}
        \end{itemize}
        \item Termination with polymer.
        \begin{itemize}
            \item We can get branched structures that are fairly stable and don't react further.
        \end{itemize}
        \item Termination by counterion.
        \item Termination by reactions with other compounds in the system: Solvent and impurities.
        \begin{itemize}
            \item Termination by water does occur. However, it is less important than in vinyl cationic polymerization on account of the cyclic monomer being more nucleophilic than the vinyl monomer.
            \item Termination by reaction of more reactive species existing in equilibrium with stable onium species.
        \end{itemize}
    \end{itemize}
\end{itemize}



\section{???}
\begin{itemize}
    \item \marginnote{2/8:}I was at the CCRF national scholars luncheon.
\end{itemize}




\end{document}