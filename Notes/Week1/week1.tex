\documentclass[../notes.tex]{subfiles}

\pagestyle{main}
\renewcommand{\chaptermark}[1]{\markboth{\chaptername\ \thechapter\ (#1)}{}}

\begin{document}




\chapter{???}
\section{???}
\begin{itemize}
    \item \marginnote{1/4:}Announcements.
    \begin{itemize}
        \item TA: Max Weires (\href{mailto:mweires@uchicago.edu}{mweires@uchicago.edu}).
        \item Talks slowly enough.
        \item This first lectures is a catchall to get everyone set.
        \item This course is cross-listed MENG 25110/35110 and CHEM 39100.
        \item There are handouts on Canvas. Ask to be added!
        \item HW will be given out by Max; it's not graded, but it's critically important that (a) we do it and (b) we understand it. Exam questions are very much like the HW.
        \item 1 HW question will appear in the Midterm and Final, so if we know the HW well, free marks!
        \item Scheduled discussion section at 11:30?? Max will schedule a second one.
        \item The entire grade is based on the two exams.
        \item Feel free to interrupt him if needed.
        \item Rowan's contact info.
        \begin{itemize}
            \item hi
        \end{itemize}
        \item Use a good book
        \begin{itemize}
            \item Odian's book is much more detailed than we need, but it's a great reference.
            \item The Stevens book is a better introductory book if this is our first brush with the subject.
        \end{itemize}
        \item Ask him to publish the slides before lecture.
    \end{itemize}
    \item Schedule.
    \emph{table}
    \begin{itemize}
        \item Two online classes while Rowan is traveling.
        \item Midterm: 2/1.
        \item Final: During finals week.
    \end{itemize}
    \item Point of this class: I give you a polymer structure, and you tell me how to make it.
    \item Example polymers.
    \begin{itemize}
        \item Polycarbonate.
        \item Poly...
        \item Kevlar
        \item PVC.
        \item PP.
    \end{itemize}
    \item By the end of this class, we should be able to make all of these.
    \item What do we need to consider when making a polymer?
    \begin{itemize}
        \item Polymer size: The physical properties of a polymer change with its size.
        \begin{itemize}
            \item Hugely important; don't let anyone tell you otherwise.
            \item Polymer synthesis means nothing unless you can (a) measure its MW and (b) control it.
        \end{itemize}
        \item Polydispersity or dispersity (PD or "D stroke", IPUAC preferred): PDI - Polydispersity index.
        \begin{itemize}
            \item Polymers \emph{always} have a distribution of MWs.
        \end{itemize}
        \item Chemistry of the polymers.
        \begin{itemize}
            \item What atoms are in it; determines synthesis.
        \end{itemize}
        \item General structure: Linear, branched, star, etc.
        \item Are there potential isomers (think copolymers; proteins are copolymers with 20 amino acids!).
        \item Control of sequence in copolymers: Random, Block, etc.
        \item What are the chain ends?
        \begin{itemize}
            \item Can they help us do more chemistry? If they're well-defined, they may be useful; varied, less so.
        \end{itemize}
        \item Rate of reaction (can it be controlled).
        \begin{itemize}
            \item Is this done very quickly, or does it take 20 days.
        \end{itemize}
    \end{itemize}
    \item The above considerations are questions that should be asked of any polymer synthesis...
    \item Note: The objective of the course is to understand the variety of tools that chemists have at their disposal to synthesize polymers.
    \item \textbf{Polymer molecules}: Formed by the reaction of many monomer molecules, which add to each other. The resulting chain molecules exhibit new properties.
    \item Effect of molecular weight in polyethylene.
    \emph{table}
    \begin{itemize}
        \item Molecular weights in the millions can stop a bullet.
        \item Implication: The chemical structure is no longer enough to tell you all properties.
    \end{itemize}
    \item How do we define the molecular weight? Degree of polymerization and MW.
    \emph{picture; polymer parts}
    \begin{itemize}
        \item The physical properties change with the length of chains.
        \item Averages are used, since a polymer material consists of many chain molecules of different length.
        \item \textbf{Structural unit} = \textbf{monomer unit} = residue of monomers A,B... incorporated in the polymer chain.
        \item \textbf{Repeating unit} = segment of one or more structural units.
        \item PCL given as an example!
        \item What people get wrong on the exam: Bracket placement. Be very careful about identifying the repeating unit!!
        \item In the left case, $DP=n$, but in the right case, $DP=2n$.
    \end{itemize}
    \item \textbf{Number-average degree of polymerization}: The average number of structural units per polymer chain. \emph{Denoted by} $\bm{\overline{X_n}}$, $\bm{\overline{DP}}$.
    \begin{itemize}
        \item The overbar means "average."
    \end{itemize}
    \item Polymer size or chain length.
    \begin{itemize}
        \item The physical properties of a polymer change with its size.
        \item So, we need to know the polymer size or molecular weight.
        \item Two most important:
        \item \textbf{Number average molecular weight}
        \begin{equation*}
            \overline{M_n} = \frac{\sum N_xM_x}{\sum N_x}
        \end{equation*}
        \begin{itemize}
            \item Total weight of molecules divided by total number of molecules.
        \end{itemize}
        \item \textbf{Weight average molecular weight}
        \begin{equation*}
            \overline{M_w} = \frac{\sum W_xM_x}{\sum W_x}
            = \frac{\sum N_xM_x^2}{\sum N_xM_x}
        \end{equation*}
        \begin{itemize}
            \item Total weight of molecules divided by total weight of molecules??
        \end{itemize}
        \item \textbf{Degree of polymerization}
        \begin{equation*}
            \overline{X_n} = \frac{\overline{M_n}}{M_\text{repeat unit}}
        \end{equation*}
        \item We do it two different ways because some experimental techniques give us $\overline{M_n}$, some give us $\overline{M_w}$, and some give us both.
        \item \textbf{Carotherts Equation}
        \begin{equation*}
            \overline{X_n} = \frac{1}{1-p}
        \end{equation*}
        where $p$ is the extent of the reaction (or fraction of conversion).
        \begin{itemize}
            \item We'll come back to this equation over and over agin.
        \end{itemize}
    \end{itemize}
    \item Measuring average molecular weights.
    \begin{itemize}
        \item Techniques which measure properties that depend on the number of molecules give $M_n$.
        \begin{itemize}
            \item End group analysis (e.g., with NMR).
            \item Osmometry.
        \end{itemize}
        \item Techniques which measure properties that depend on the size or mass of the molecules give $M_w$.
        \begin{itemize}
            \item Light Scattering.
        \end{itemize}
        \item Other techniques are relative, i.e., properties compared to known standars.
        \begin{itemize}
            \item Gel electrophoresis.
            \item ...
        \end{itemize}
    \end{itemize}
    \item Dispersity.
    \emph{table}
    \begin{itemize}
        \item In synthetic polymers, the individual chain molecules are not the same weight.
        \item A measure of the breadth of distribution of the molecula weights in a polymer is the dispersity (used to be termed the PDI):
        \begin{equation*}
            D = \frac{\overline{M_w}}{\overline{M_n}} > 1
        \end{equation*}
        \item Hypothetical single MWt polymer has $D=1.000$. Could be made with a tedious 20-step synthesis, but Rowan doesn't recommend that.
        \item We'll talk a bit about living polymerizations.
        \item Radical often gives 2, but can be 1.5.
        \item Addition and step-growth (we'll talk more about them in a minute) give 2.0.
        \begin{itemize}
            \item The chemistry has to be almost quantitative (99.9\% if we wanna use it). At 90\%, we're already at PDI of 10.
            \item Technical term: Shitty polymers, which are basically just oils that do nothing to us.
            \item Almost all reactions have side reactions that kill termination.
        \end{itemize}
        \item High-conversion polymers have things dying often and the PDI grows.
    \end{itemize}
    \item The chemistry: The two fundamental mechanisms for polymerization.
    \begin{itemize}
        \item \textbf{Chain-growth}: Addition polymerization. E.g., Olefins, some ring openings.
        \item \textbf{Step-growth}: Condensation polymerization. E.g., polyesters, polyamides.
        \item Not perfect correspondence, though: Polyurethanes use step-growth, not condensation for some reason.
        \begin{equation*}
            \ce{OCN-R-NCO + HO-R$'$-OH -> Polyurethane}
        \end{equation*}
        \item Why distinguish between chain-growth and step-growth?
        \begin{itemize}
            \item The mechanisms and setups are quite different.
        \end{itemize}
    \end{itemize}
    \item Chain (addition) polymerizations.
    \emph{picture}
    \begin{itemize}
        \item \ce{R^*} + Monomer \ce{->} \ce{R-M^*}
        \item And then repeat.
        \item So a few initiatiors and a lot of polymers lead to fast buildup of molecular weight.
        \item Only monomer, polymer, and a small number of growing chains present at any one time.
        \begin{itemize}
            \item This is key!
            \item New car smell: We're getting high on the monomer, which wasn't fully removed from the polymers; the monomers are more volatile, hence us smelling it!
            \item We have monomer at any stage of reaction, except we go to 100\% completion, but this almost never happens because of viscosity increasing.
        \end{itemize}
        \item All that happens as we to through time is we get more polymer chains; MW doesn't really vary throughout the process.
    \end{itemize}
    \item Step-growth polymerization.
    \emph{picture}
    \begin{itemize}
        \item Any two components can add together to increase molecular weight.
        \item Slow buildup of MW.
        \item High conversions required to get high MW.
        \begin{itemize}
            \item This is doable, but it takes some skill.
            \item Much easier to do chain growth.
        \end{itemize}
        \item If we made a car with step-growth, no new car smell! Monomer disappears very quickly.
    \end{itemize}
    \item Comparisons of chain- and step-growth polymerizations.
    \begin{table}[h!]
        \centering
        \begin{tabular}{|c|c|}
            \textbf{Step} & \textbf{Chain}\\
            \hline
            Growth by the reaction of any two oligomers & Growth by addition of monomer only at the end of one chain\\
            Rapid loss of monomer species (First reaction produces dimer) & Some monomer remains even after long reaction times.\\
            Driving force for the reaction is removal of the condensation product & Driving force for thie reaction energetics, i.e., conversion of 1 double bond into 2 single bonds (Addition)\\
            Molar mass increases slowly throughout & Molar mass of backbone increases rapidly\\
            Ends remain active & Chain not active after termination\\
            ...
        \end{tabular}
        \caption{Comparison of chain- and step-growth polymerizations.}
        \label{tab:ChainVsStep}
    \end{table}
    \begin{itemize}
        \item This has been a favorite question in the past.
    \end{itemize}
    \item Double-bond polymerizations.
    \begin{itemize}
        \item This is really the focus of the first half of the course.
        \item Driving force for the reaction is the conversion of 1 $\pi$-bond (ca. \SI[per-mode=symbol]{585}{\kilo\joule\per\mole} to 2 $\sigma$-bonds (2x ca. \SI[per-mode=symbol]{330}{\kilo\joule\per\mole})).
        \item The overall polymerization generallly involves 4 basic reactions: Initiation, propagation, termination, and chain transfer.
    \end{itemize}
    \item \textbf{Initiation}: 
    \item \textbf{Termination}: \emph{Also known as} \textbf{death}.
    \item \textbf{Chain transfer}: New reactive species formed at same time growing polymer chain is deactivated (terminated).
    \item Next few slides are for the engineers: There are only 3-4 OChem-era reaction mechanisms, which we'll go over right now; chem majors can yawn.
    \begin{itemize}
        \item Most reactions involve an electron-rich molecule (\textbf{nucleophile}) and an electron-poor molecule (electrophile).
        \begin{itemize}
            \item Rowan "is VERY PICKY with curved arrow formalism."
        \end{itemize}
        \item Radical reactions involve the coupling of two unpaired electrons, or can attack a double bond.
        \begin{itemize}
            \item A lot of polymerization involves radical chemistry.
        \end{itemize}
        \item A reactive carbon center (\ce{R^*}) can exist as a electrophile (carbenium ion, carboncation), carbon radical, or nucleophile (carbanion).
        \item Only electron-poor double bonds will react with the anion. Only nucleophilic double bonds will react with carbocations.
        \begin{itemize}
            \item Implication: Not every double bond can be polymerized the same way!
        \end{itemize}
        \item The reactive species can be a free radical (neutral reactive species [RS]), cationic (RS is positive and electrophilic), and anionic [reactive species is negative and nucleophilic].
        \begin{itemize}
            \item So a common questions is, given a certain kind of bond, what kind of initiator would you use?
        \end{itemize}
        \item The reactive species depends on the substituents on the molecule.
        \begin{itemize}
            \item A bit of electron withdrawing often helps polymerization.
            \item Radical: Requires X,Y to be EWG (to stabilize the radical).
            \item Cationic: Requires X,Y to be EDGs (to stabilize the electron-poor carbenium ion).
            \item Anionic: Requires X,Y to be EWGs (to stabilize the electron-rich carbanion).
        \end{itemize}
    \end{itemize}
    \item Some monomers and how they can be polymerized.
    \emph{table}
    \item How to tell if a substituent is EDG or EWG.
    \begin{itemize}
        \item Electron density from the O lone pairs donate into the $\pi$-bond. Other heteroatoms with a lone pair can also donate electrons in this manner. Alkyl groups can also "donate" electron density to the double bond, but it is much weaker.
        \item EWGs: Electron densituy from the $\pi$-bond is withdrawn by the carbonyl into the ester. Most substituents that have a multiple bond adjacent to the doulbe bond will withdraw electrons, e.g., CN. Halides are also EWGs.
        \item Key point: Any time you can delocalize electron density, that makes the system more stable.
        \begin{itemize}
            \item If you can share electron density across many atoms, that dramatically increases stability.
        \end{itemize}
    \end{itemize}
    \item Styrene derivatives; The effect of resonance.
    \begin{itemize}
        \item "If you look at it the wrong way, styrene will polymerize."
        \item It's very reactive and can follow any of the mechanisms: Radical, cations, and anions can all be shared among 4 carbons!
        \item It's not magic why certain initiators work; it's chemistry!
        \item "Last thing you want to do is spend a month of your PhD trying to make a polymer only to realize you used the wrong initiating system."
    \end{itemize}
    \item Insertion "coordination" polymerization.
    \begin{itemize}
        \item In this type of polymerization, the metal is coordinated to the chain end.
        \item It also coordinates to the monomer, in the process of activiating it, before inserting the monomer between the chain end and metal.
        \item Ziegler-Natta and ...
        \begin{itemize}
            \item We'll have a class on these.
        \end{itemize}
    \end{itemize}
    \item Ring-opening polymerizations (ROP).
    \begin{itemize}
        \item Thermodynamic driving force here is the release of ring strain.
        \item This is how polyethylene oxide is made.
        \item ROP usually requires an initiator, c.f., vinyl additio polymerizations.
        \item ROP can be initiated by either anionic or cationic polymerizations depending on the monomer.
        \item Polymerization, generally, by addition of monomer only to reactive chain end, c.f., vinyl additio polymerizations.
    \end{itemize}
    \item Living polymerizations.
    \begin{itemize}
        \item Chain polymerizations where chain transfer and termination are either very small or nonexistent and all chains are initiated at once.
        \item Generally step-growth (Rowan knows no chain-growth examples).
        \item There's no chain death here; without death, the system is \emph{living}!
        \item Normal chain growth: With time, number of polymer chains increases and $\overline{M_n}$ stays the same.
        \item Living growth: Same number of polymer chains, but $\overline{M_n}$ increases.
        \item Living polymerizations are a great way to control molecular weight! If you want a certain MW, just stop after a certain amount of time.
        \item Kinetics: $R_t=0$; no termination. $R_i\gg R_p$: Rate of initiation must be faster than rate of propagation so that all chains start growing at once and they all increase MW at the same time.
        \item If we have these two kinetic conditinos, then
        \begin{equation*}
            X_n = \frac{[M]_\text{consumed}}{[I]}
        \end{equation*}
        and $D\approx 1$.
    \end{itemize}
    \item Step-growth polymerization.
    \begin{itemize}
        \item Esterification.
        \begin{itemize}
            \item Acid chloride plus alcohol makes ester and HCl.
            \item This is condensation.
        \end{itemize}
        \item Polyesterification.
        \begin{itemize}
            \item Just take a di-acid chloride and a diol and condense!
            \item \ce{HCl} isn't a great nucleophile, but it can still reverse the polymerization.
            \item Thus, to get high MW, you want to withdraw the condensation product.
        \end{itemize}
    \end{itemize}
    \item Types of polymers formed by step-growth polymerizations.
    \emph{picture with 6 subfigures}
    \begin{itemize}
        \item Technically, "polyimine" should be "polyamine," but whatever says Rowan.
    \end{itemize}
    \item General mechanisms: Substitution reactions.
    \begin{itemize}
        \item SN2, two molecules involved in the RDS.
        \item SN1, one molecule involved in the RDS.
    \end{itemize}
    \item General mechanism: acyl substitution.
    \begin{itemize}
        \item Polyester or polyamide formation.
        \item Remember the famous/infamous tetrahedral intermediate!
        \item Rate depends on how good the LG is; chloride is a great leaving group, which is why acid chlorides are so reactive!
        \item Note.
        \begin{itemize}
            \item This is a substitution reaction; however, the mechanism is slightly different from a normal SN2 mechanism.
            \item The electrophile can be an acid chloride, acid anhydride, ester, or just an acid, although the reaction is slower the worse the leaving group X is.
            \begin{itemize}
                \item Acids need like \SI{200}{\celsius}.
            \end{itemize}
        \end{itemize}
    \end{itemize}
    \item Comparision of chain- and step-growth polymerizations.
    \begin{itemize}
        \item Chain-growth: Slow initiation, fast propagation yields increasing number of chains with same weight.
        \begin{itemize}
            \item $p=R_p/(R_p+R_t+R_{ct})$ from the Carothers Equation, where $R_p$ is rate of propagation, $R_t$ is rate of termination, and $R_{ct}$ is rate of chain transfer.
        \end{itemize}
        \item Step-growth: Exponential increase in weight toward the end; needs high conversion! Essentially, weight follows the equation $2^n$: Monomers to dimers to tetramers to octomers to 16-mers, on and on.
        \begin{itemize}
            \item $p$ is percent of functional groups reacted divided by 100.
        \end{itemize}
    \end{itemize}
    \item Possible isomers in polymers.
    \begin{itemize}
        \item Structural isomers.
        \item Sequence isomers.
        \item Stereoisomers.
        \item Geometric isomers.
        \item Regioisomers.
    \end{itemize}
    \item \textbf{Structural isomers}.
    \begin{itemize}
        \item Linear vs. branched.
        \item Branched polymers have very different physical properties to linear polymers; less crystalline...
    \end{itemize}
    \item Sequence isomerism.
    \begin{itemize}
        \item There are generally two possible ways in which a monomer can add to a reactive site.
        \item Most olefinic monomers give almost exclusively HD polymers, mainly on account of steric hindrance and stability of the active site.
        \item However, some polymers (e.g, PVF) have a significant amount of HH and TT placements.
        \item ?? could be the weakest link in our polymers.
    \end{itemize}
    \item Stereoisomers.
    \begin{itemize}
        \item Stereochemistry of addition polymerization of olefins.
        \item The two faces of a double bond or reactive center -- prochirality.
        \item So for any polymerization of an unsymmetrical double bond, there are another two possible ways for the monomer to be added to the polymer chain.
    \end{itemize}
    \item Tacticity.
    \begin{itemize}
        \item Stereoisomerism plays a major role in controlling the polymer's ability to crystallize.
        \item Three ggeneral types: Isotactic, syndiontactic, atactic.
        \begin{itemize}
            \item Remember by drawing letters I,S through the polymer shape!
        \end{itemize}
    \end{itemize}
    \item Dyad tacticity is defined to be the fraction of pairs of adjacent units that are either isotactic or syndiotactic to one another.
    \begin{itemize}
        \item Triads exist, too!
    \end{itemize}
    \item Random tacticity.
    \item Radical polymerization.
    \begin{itemize}
        \item When we do a polymerization, we'll generally...
    \end{itemize}
\end{itemize}




\end{document}