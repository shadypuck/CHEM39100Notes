\documentclass[../notes.tex]{subfiles}

\pagestyle{main}
\renewcommand{\chaptermark}[1]{\markboth{\chaptername\ \thechapter\ (#1)}{}}
\setcounter{chapter}{7}

\begin{document}




\chapter{???}
\section{???}
\begin{itemize}
    \item \marginnote{2/20:}Relation of reactivity to chemical structure.
    \begin{itemize}
        \item Polar effects: Alternation.
        \item Enhanced reactivities are observed between certain pairs of monomers on account of subtle radical monomer interactions, which can lead to alternating tendency in such polymerizations.
        \item This is observed between copolymerization of electron rich and electron poor double bonds.
        \begin{itemize}
            \item Example: Styrene (electron-rich) and maleic anhydride (electron-poor).
        \end{itemize}
    \end{itemize}
    \item Reactivity ratios.
    \begin{itemize}
        \item Many ways to visually represent these, such as \textbf{reactivity triangles}.
        \item Shows that as you get greater difference between electron rich and electro poor, $r_1r_2$ approaches 0.
    \end{itemize}
    \item Reactivity ratios summary.
    \begin{itemize}
        \item A great visual description of everything discussed last time.
        \item Very concise, too!
        \item Emphasis: "Random" and "statistical" are \emph{not} interchangeable!
    \end{itemize}
    \item Composition drift.
    \begin{itemize}
        \item The copolymerization equtions give the instantaneous copolymer composition.
        \item However, in all copolymerizations except \textbf{azeotropic copolymerizations}, the comonomer feed changes in composition as one of the monomers gets incorporated preferentially into the polymer. This leads to a drift in monomer composition to the less reactive monomer.
        \item There's a kind of hexagonal (view it three-dimensionally!) chart that summarizes all this.
    \end{itemize}
    \item \textbf{Azeotropic} (copolymerization): The formation of a copolymer for which the overall composition of copolymer monomeric units equal to the composition of the feed.
    \item Summary.
    \begin{itemize}
        \item The key parameters in copolymerization are the reactivity ratios, which influence the relative rates at which a given radical will add to the same monomer vs. a comonomer.
        \item The copolymerization equation relates the mole fraction of monomers in the polymer to the composition of the feedstock via the reactivity ratios.
        \begin{itemize}
            \item Different classes of behavior may be assigned based on the product of the reactivity ratios...
        \end{itemize}
        \item Statistical considerations give predictions for the average sequence length and sequence length distributions in a copolymer on the basis of reactivity ratios and feedstock composition.
    \end{itemize}
    \item That's it for the last topic.
    \item Now it's onto step-growth polymerizations.
    \item Step-growth polymerizations: An introduction.
    \begin{itemize}
        \item This is a review of the intro slides.
        \item The invention of step polymerization is often credited to Wallace Hume Carothers and Julian Hill.
        \item They prepared polyesters from propylene glycol and hexadecanedioic acid to make artificial silk fibers (1929 at DuPont).
        \item Conclusions.
        \begin{itemize}
            \item Difficult to produce high molecular weight polymers.
            \item $M_n$ initially limited to \numrange{1500}{4000}; acid catalyst increases $M_n$.
            \item Equilibrium reaction; remove water to increase $M_n$ to 12,000.
            \item Conclude that...
            \item "Atractive appearance" is a direct quote --- old chemists were very descriptive!.
            \item ...
            \item From a JACS paper in 1929.
        \end{itemize}
    \end{itemize}
    \item Some commercial step-growth polymers.
    \begin{itemize}
        \item Polyester, nylon, Kevlar, polyurethane, etc.
    \end{itemize}
    \item Step-growth polymerization: Introduction.
    \begin{itemize}
        \item Di-acids and di-amides.
        \item In many step-growth reactions, a low-molecular weight byproduct (\ce{X}) is prouduced.
        \item Because the reactions are equilibriums, the byproduct must be removed to shift the equilibrium to the product side and create high-molecular weight products.
        \item Since the byproduct is often removed by condensation (\ce{H2O}, \ce{MeOH}, \dots), the reaction is termed \textbf{polycondensation}.
        \item Thus, for historical reasons, the terms step-growth polymerization and polycondensation are often used synonomously, althoughthis is not always correct.
        \begin{itemize}
            \item Important exception: Urethanes do not produce any condensation product.
        \end{itemize}
    \end{itemize}
    \item Step-growth polymerization: General mechanism.
    \begin{itemize}
        \item Mechanism: Oligomers combine.
        \item Rapid disappearance of monomer species.
        \begin{itemize}
            \item Makes these harder to track by NMR.
        \end{itemize}
        \item Reactivity of functional groups is independent of chain length.
        \item Molecular weight increases slowly.
        \item No initiator (although catalyst can help kinetics).
        \begin{itemize}
            \item This makes the kinetics a bit easier actually.
        \end{itemize}
        \item Chain ends are still active.
    \end{itemize}
    \item Molecular weight.
    \begin{itemize}
        \item What is a polymer?
        \begin{itemize}
            \item Controversial question.
            \item Rowan says when you get above the entanglement rate.
        \end{itemize}
        \item Practical consideration.
        \begin{itemize}
            \item Only polyners with $M_n$ of $>\SI{10000}{\gram\per\mole}$ have useful mechanical properties.
            \item Note: Electronic polymers may be useful at lower $M_n$.
        \end{itemize}
        \item \textbf{Structural unit} = \textbf{monomer unit} = \textbf{residue}, e.g., a diol or diacid..
        \item \textbf{Repeat unit}: A segment of one or more structural units that repeats along the polymer chain.
        \item The \textbf{number-averaged ...}
    \end{itemize}
    \item Number average degree of polymerization.
    \begin{itemize}
        \item Can be simply defined as the total number of molecules originally present in the system $N_0$ divided by the total number of molecules in the system after the polymerization $N$.
        \item If [\ce{M}] is the concentration of molecules, then
        \begin{equation*}
            \overline{X}_n = \frac{N_0}{N}
            = \frac{[\ce{M}]_0}{[\ce{M}]}
        \end{equation*}
        \item $X_n$ can also be related to the concentration of one of the functional groups [\ce{M}] present after a fraction $p$ has been reacted, where $[\ce{M}]_0$ is the initial concentration of one of the functional group.
        \item A bit more stuff gets you back to \textbf{Carothers Equation}, proposed by Carothers in 1936.
        \begin{equation*}
            \overline{X}_n = \frac{1}{1-p}
        \end{equation*}
        \begin{itemize}
            \item This emphasizes the need for nearly quantitative chemistry once again (i.e., chemistry for which $p\to 1$).
        \end{itemize}
    \end{itemize}
    \item Number-average molecular weight.
    \begin{itemize}
        \item Defined as the total weight of a polymer sample divided by the total number of moles in it.
        \item Given by
        \begin{equation*}
            \overline{M}_n = \frac{\sum N_xM_x}{\sum N_x}
            = M_0\overline{X}_n+M_\text{eg}
            = \frac{M_0}{1-p}+M_\text{eg}
        \end{equation*}
        \item It is important to note for type AA + BB polymerizations, the repeat units contain bits of both monomers. Therefore, $M_0$ is the \textbf{mean molecular weight} of the two structural units (see below)...
    \end{itemize}
    \item What is the impact of the Carothers equation?
    \begin{itemize}
        \item Step-growth polymerization is challenging.
        \item If your chemistry isn't perfect and your reaction isn't driving you to high conversions, it's not going to work.
    \end{itemize}
    \item Molecular weight and stoichiometric.
    \begin{itemize}
        \item The reactant ratio $r$ (\emph{not to be confused} with the reactivity ratio; horrible choice of notation).
        \item For non-stoichiometric monomer ratios, the \textbf{reactant ratio} $r$ is introduced.
        \begin{equation*}
            r = \frac{N_{\ce{A}}}{N_{\ce{B}}}
        \end{equation*}
        \begin{itemize}
            \item Note that $r\leq 1$.
            \item $N_{\ce{A}},N_{\ce{B}}$ are the numbers of functional groups \ce{A} and \ce{B} initially present.
        \end{itemize}
        \item The total number $N_0$ of molecules initially present becomes
        \begin{equation*}
            N_0 = \frac{N_{\ce{A}}+N_{\ce{B}}}{2}
            = ...
        \end{equation*}
        \item ...
        \item We get to
        \begin{equation*}
            \overline{X}_n = \frac{1+r}{1+r-2rp}
        \end{equation*}
        \item Table.
        \item Chart.
    \end{itemize}
    \item Molecular weight control.
    \begin{itemize}
        \item ...
        \item Since in step polymerization, DP (degree of polymerization) is a function of reaction time, it can --- in principle --- be controlled by the latter (i.e., the reaction can be quenched once the desired molecular weight is obtained).
    \end{itemize}
    \item Molecular weight and stoichiometric.
    \begin{itemize}
        \item Case of an AA/BB polymerization.
        \begin{itemize}
            \item Monomer BB is used in excess.
        \end{itemize}
        \item End cappers case.
        \begin{itemize}
            \item ...
            \item Here, we have
            \begin{equation*}
                r = \frac{N_{\ce{A}}}{N_{\ce{B}}+2N_{\ce{B}}'}
            \end{equation*}
        \end{itemize}
        \item Type 3: For an \ce{A-B} monomer, the $r$ is automatically 1.
    \end{itemize}
    \item Molecular weight distribution.
    \begin{itemize}
        \item Most probable dispersity is 2.
        \item Breadth of molecular weight distribution.
        \item We also have the weight-averaged degree of polymerization.
    \end{itemize}
    \item General considerations on step-growth polymerization chemistry.
    \begin{itemize}
        \item High degree of polymerization only at high conversions.
    \end{itemize}
    \item Basic chemistry of carboxylic acid derivatives.
    \begin{itemize}
        \item Functional group and type of polymer.
        \item To polymerize these guys, all we need is OChem II.
        \item We heat the monomers to \SI{200}{\celsius} under vacuum and just drive off the water.
        \begin{itemize}
            \item This is super easy to do at large scale.
            \item This is why nylon is produced commercially: We can just heat it up really hot and in bulk batches.
        \end{itemize}
        \item Polycarbonates: Alcohol plus phosgene.
        \item Polyurethanes: Alcohol plus isocyanate.
        \item We can utilize all of these chemistries!
    \end{itemize}
\end{itemize}



\section{???}
\begin{itemize}
    \item \marginnote{2/22:}Final is 3/5, 10-12 in this room.
    \item Last time, we started talking about chain-growth polymerizations. What do we remember?
    \begin{itemize}
        \item You want your stoichiometry to be as close to one as possible.
        \item We want things to be nearly quantitative.
        \item Works well in large batch.
        \item Sometimes we need a catalyst; initiators aren't needed, though.
        \item We need to drive the polymerizations by removing the condensation product.
        \item Most recyclable commercial plastic right now is PET (\underline{p}oly\underline{e}thylene\underline{t}ethalate).
        \begin{itemize}
            \item We can \emph{chemically} recycle this since it has an ester backbone. (vs. \emph{mechanical} recycling)
            \item This is fairly new...
        \end{itemize}
    \end{itemize}
    \item Basic chemistry of carboxylic acid derivatives.
    \begin{itemize}
        \item This mechanism is similar for polyester, polyamide, polyimide, etc.
        \item An acid-catalyzed process adds an acid to the carbonyl. This makes the carbon even more electrophylic. We get to the tetrahedral intermediate, and then electrons kick down and kick out hydroxide/water.
        \item Every step is completely reversible and we have to account for this when we think about the kinetics.
        \item Removing the water uses Le Ch\^{a}telier's Principle to drive the equilibrium to the right.
        \item The rate-limiting step is the alcohol/amine reacting with the activated acid ($k_3$).
    \end{itemize}
    \item Kinetics of the polyesterification of diacid and diol.
    \begin{itemize}
        \item Equal reactivity of the functional groups on each end of the monomer.
        \item The reactivity of the FGs is largely independent of chain length.
        \item Reaction state can be monitored by looking at how many intact end groups you have left.
        \item The rate of reaction is expressed in terms of the concentrations of the reacting FGs.
        \item The rate of polyesterification:
        \begin{equation*}
            R_p = -\dv{[\ce{R-COOH}]}{t} = ...
        \end{equation*}
        \item ...
        \item $[\ce{R-C(OH)2+}]$ is difficult to measure. However,
        \begin{equation*}
            K = \frac{[\ce{R-C(OH)2+}]}{[\ce{R-COOH}][\ce{HA}]}
        \end{equation*}
        \item Therefore,
        \begin{equation*}
            R_p = ...
        \end{equation*}
        \item Now we can also have a self-catalyzed polymerization.
        \begin{itemize}
            \item In the absence of additional strong acid, \ce{R-COOH} assumes the role of \ce{HA}.
        \end{itemize}
        \item Thus, here,
        \begin{equation*}
            R_p = -\dv{[\ce{R-COOH}]}{t} = k[\ce{R-COOH}]^2[\ce{R-OH}]
        \end{equation*}
        \begin{itemize}
            \item The experimentally determined rate constant $k$ in this equation equals $Kk_3$.
            \item This is a third-order reaction.
        \end{itemize}
        \item ...
        \item In terms of the \textbf{extent} or \textbf{fraction of conversion}
        \begin{equation*}
            p = \frac{[\ce{M}]_0-[\ce{M}]}{[\ce{M}]_0}
        \end{equation*}
        we have that
        \begin{align*}
            2kt &= \frac{1}{[\ce{M}]^2}-\frac{1}{[\ce{M}]_0^2}\\
            & ...
        \end{align*}
    \end{itemize}
    \item Accessibility of functional groups.
    \begin{itemize}
        \item To yield a high-MW polymer, the polymer must not precipitate from the polymerization mixture before the desired molecular weight is reached.
        \item Often tricky due to the limited choice of solvents that dissolve the polymer and are compatible with the reaction conditions.
        \item In xylene, the polymer precipitats at once.
        \item In nitrobenzene, the polymer precipitates after \SI{30}{\minute}.
        \item In DMSO, the polymer is directly soluble.
    \end{itemize}
    \item Equilibrium considerations.
    \begin{itemize}
        \item Most step polymerizations involve equilibrium reactions. Thus, it is important to elucidate how the equilibrium affects the extent of conversion. Compare closed system with open system.
    \end{itemize}
    \item Equilibrium conditions (closed).
    \begin{itemize}
        \item The polymer and condensation products build up until the rate of the reverse reaction (depolymerization) equals that of the forward.
        \item The equilibrium constant equals
        \begin{equation*}
            K = \frac{p^2}{(1-p)^2}
        \end{equation*}
        \item It follows that
        \begin{equation*}
            p = \frac{\sqrt{K}}{1+\sqrt{K}}
        \end{equation*}
        \item This tells you that your equilibrium constant basically has to be in the hundreds of thousands or millions to get anything resembling polymer.
        \item Esterification equilibrium constant is somewhere between 1 and 10, so it definitely has to be driven.
        \begin{itemize}
            \item Amides are 100-1000.
            \item In a closed system, you'll never defeat the laws of thermodynamics.
        \end{itemize}
        \item We get
        \begin{equation*}
            \overline{X}_n = 1+K^{1/2}
        \end{equation*}
    \end{itemize}
    \item Equilibrium conditions (open).
    \begin{itemize}
        \item High MWT polymers require an open, driven system: At least one of the products of the forward reaction must be removed to drive the equilibrium towards high molecular weights.
        \item Remove volatile, low-molecular weight byproducts such as water, methanol, etc. with temperature, reduced pressure, and purging with an inert gas (e.g., nitrogen or argon).
        \item HCl can be removed by adding a base to neutralize the acid once created.
        \item Sometimes the product is eliminated from the reaction because it precipitates.
        \item You can get diffusion control as viscocity skyrockets; this is how you get bubbles in your polymer.
        \item We get
        \begin{equation*}
            \overline{X}_n = \left( \frac{K[\ce{M}]_0}{[\ce{H2O}]} \right)^{1/2}
        \end{equation*}
        \item Thus, even with a crappy $K$, we can get high molecular weight with water removal.
        \item We don't need to remove all of the water.
        \item There are engineering approaches to removing water and preventing bubbles, such as thing sheets and curing, running the system in an emulsion to aid SA:Volume ratio, etc. It doesn't all have to be in a big vat!
    \end{itemize}
    \item Polymerization vs. cyclization.
    \begin{itemize}
        \item A polymer can react intramolecularly and close up into a ring.
        \item In most any chain-growth polymerization, you will get some competitive ring formation.
        \item You do have to pay an entropy penalty to bring the two chain ends close together, but it will still happen in some equilibrium.
        \item Ring formation is usually undesired.
        \item Whether for a particular system ring-formation is competitive with lienar polymerization depends on both thermodynamics and kinetic considerations.
        \item We don't often form small rings; if you're designing a polymerization and you know that six-membered rings could form, that's not one you'd choose to run because you'll just be fighting against that the entire time.
    \end{itemize}
    \item Kinetics of cyclization.
    \begin{itemize}
        \item Kinetic feasibility.
        \item The kinetic feasibility of ring formation depends on the probability of two groups of the reactant molecule to approach each other to react.
        \item As the potential ring size increases, the molecule...
        \item Kinetic feasibility decreases with ring size.
        \begin{itemize}
            \item You need a good solvent that swells the chain and makes it stretch out rather than bunch.
        \end{itemize}
    \end{itemize}
    \item Tendency of cyclization.
    \begin{itemize}
        \item The easiest ring to form kinetically is three. It's terrible thermodynamically, but it's kinetically great.
        \item Thus, we've got a balance.
        \item The effect of cyclization is also counterbalanced by the concentration factor; high concentration = fewer cycles; low concentration = more cycles.
        \item Many commercial processes remove low molecular cyclics by estraction or devolatization or boiling them off at really high temperatures.
        \item Early in the chart, it's kinetically feasible.
        \item Later on, ring strain is no longer a problem and we're entirely in the realm of kinetics.
    \end{itemize}
    \item Interchange reactions.
    \begin{itemize}
        \item Some polymers (polyesters, polyamides, etc.) can --- under appropriate conditions...
    \end{itemize}
    \item Step-growth copolymers.
    \begin{itemize}
        \item Homopolymers are easy.
        \item Copolymers; we can get different values for the numbers of polymers.
        \item Alternatign copolymers vs. statistical and random copolymers.
    \end{itemize}
    \item Step-growth copolymer synthesis.
    \begin{itemize}
        \item Overall copolymer composition is usually the same as the feed ratio as the reaction will need to be taken to 100\%.
        \item Random copolymers are common as usually the reactivity of the functional groups in the different monomers of the same size.
    \end{itemize}
    \item So that's all the chemistry.
    \item Now, we'll talk about some specific polymers.
    \item PET.
    \begin{itemize}
        \item Poly(ethylene terephthalate).
        \item Anything that says polyethylene on our clothes.
        \item Coke bottles.
        \item The most important commercial polymer.
        \item Aka Mylar, Dacron, Terylene.
        \item Melting point at \SI{255}{\celsius}.
        \item Glassy transition temperature at \SI{80}{\celsius}.
        \item Optically clear.
        \item Potentially high crystallinity.
        \item High density.
        \item Cheap.
        \item Good mechanical properties.
        \item Good chemical resistance (labile in strong acids and bases).
        \item Low dye-ability.
        \item Fibers.
        \begin{itemize}
            \item $M_n>\num{15000}$.
            \item Melt-spinning, etc.
            \item Use in apparel, curtians, upholstery, fishing lines, ...
        \end{itemize}
        \item Films.
        \begin{itemize}
            \item Use in: Audio and video tapes, floppy disks, capacitors.
        \end{itemize}
        \item Bottles.
        \begin{itemize}
            \item $M_n>\num{25000}$.
            \item Solid-state post condensation.
            \item Low crystallinity pre-form; stretch blow molding.
        \end{itemize}
        \item \emph{These} are not the ones that leach into water; it's the polycarbonates, such as BPA.
    \end{itemize}
    \item PET synthsis.
    \begin{itemize}
        \item Ester-interchange route.
        \item First stage.
        \begin{itemize}
            \item A mixture of dimethyl terephthalate and excess ethylene glycol (as solvent) are heated. Methanol is released as a vapor.
        \end{itemize}
        \item Second stage.
        \begin{itemize}
            \item Much hotter, lower pressure. Ethylene glycol comes off now and the polymer grows.
        \end{itemize}
        \item Second step.
        \begin{itemize}
            \item Antimony (Sb) centered polycondensation catalyst.
            \item This catalyst is the reason we can get clear Sprite bottles now.
            \item The reason we used to have green Sprite bottles is that the PET would be yellow with the old catalyst and then they would add a blue dye because no one wants to be drinking out of a pee-colored bottle.
            \item Then the chemistry evolved and now we can make clear bottles.
        \end{itemize}
        \item Polyester side reactions.
        \begin{itemize}
            \item Ethylene glycol can self-react, releasing water, to form di-ethylene glycol.
            \item We can also get dehydration.
            \item The reason that polymers start to get brown at higher temperatures is that there's some degradation reactions forming UV chromophores; pure, it should be clear, white, or colorless.
        \end{itemize}
    \end{itemize}
    \item Other polyesters.
    \begin{itemize}
        \item PETG.
        \item Aliphatic polyesters.
        \begin{itemize}
            \item Not usually as useful.
            \item Used as plasticizers for PVC.
            \item Diols can be used as building blocks for polyurethane.
            \item Poly(lactic acid) is the only commercial scale polymer made from biological sources.
            \item Liquid crystalline polymers...
        \end{itemize}
    \end{itemize}
    \item We'll start next class with BPA.
\end{itemize}




\end{document}