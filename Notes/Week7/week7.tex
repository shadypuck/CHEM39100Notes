\documentclass[../notes.tex]{subfiles}

\pagestyle{main}
\renewcommand{\chaptermark}[1]{\markboth{\chaptername\ \thechapter\ (#1)}{}}
\setcounter{chapter}{6}

\begin{document}




\chapter{???}
\section{Coordination/Insertion Polymerization}
\begin{itemize}
    \item \marginnote{2/13:}Adding a coordinating metal chain to the end of the polymer, and then the metal activates the monomer.
    \item \textbf{Coordination polymerization}: A polymerization in which the growing polymer is bound to a metal atom and that insertion of the monomer into the metal-bound polymer is preceded by, and presumably activated by, the coordination of the monomer with the metal. \emph{Also known as} \textbf{insertion polymerization}.
    \item Introduction.
    \begin{itemize}
        \item Both olefin polymerizations and ROPs can be carried out by insertion polymerization.
        \item For asymmetric monomers, there are two possible modes of insertion (primary/1,2 and secondary/2,1).
        \item It's mainly the $d$ orbitals of transition metals that can do this.
    \end{itemize}
    \item Free radical polymerization of ethylene.
    \begin{itemize}
        \item Does not result in a linear change; gives LDPE with many short chains due to backbiting.
        \item 5-15 $n$-butyl chains and 1-2 ethyl, $n$-amyl, and $n$-hexyl branches per 500 monomer units.
        \begin{itemize}
            \item Produced via backbiting.
        \end{itemize}
        \item Short chain branches outnumber the longer branches by 20-50 times.
        \begin{itemize}
            \item Produced via chain transfer to polymer.
        \end{itemize}
        \item Quite a broad dispersity (PD 3-20).
        \item Low glass transiton temperature and melts (40-60\%) at \SIrange{105}{115}{\celsius}.
        \item We \emph{cannot} make HDPE using free radical polymerization.
    \end{itemize}
    \item Free radical "polymerization" of polypropylene?
    \begin{itemize}
        \item Reacting propylene with free radicals does not yield polymers, only oligomers.
        \item We get a stable allyl radical.
        \item Some grocery store bags are LDPE (that's why they break). Good HDPE can stop a bullet.
    \end{itemize}
    \item Olefin polymerization with coordination catalysts.
    \begin{enumerate}
        \item Polymerization of ethylene.
        \begin{itemize}
            \item Ziegler-Natta catalyst produces linear polyethylene (HDPE).
            \item Ziegler and Natta won the Nobel Prize for this!
        \end{itemize}
        \item Can polymerize propylene and other $\alpha$-olefins.
        \begin{itemize}
            \item Copolymer of hex-1-ene and ethylene produces linear low-density polyethylene (LLDPE); like LDPE but requires less energy to produce.
        \end{itemize}
        \item Specific tacticity can be achieved with $\alpha$-olefins.
        \begin{itemize}
            \item Isotactic polymers.
            \item Syndiotactic polymers.
        \end{itemize}
        \item Homogeneous and heterogeneous catalysts.
        \begin{itemize}
            \item Solution phase vs. pumping stuff over a solid catalyst.
        \end{itemize}
    \end{enumerate}
    \item Work on polymerizing propylene was nearly abandoned prior to the discovery of Ziegler-Natta catalysts!
    \begin{itemize}
        \item Take away: Don't abandon ideas...
    \end{itemize}
    \item Ziegler-Natta polymerization of non-polar olefin monomers.
    \begin{itemize}
        \item The discovery was completely serendipitous.
        \item This is a really good instance of keeping an open mind; they were able to realize that what they had found was way more cool than what they were working toward.
        \item Book: \emph{The Chain Straighteners.}
        \item They cleaned the bomb really well, but they \emph{needed} some of the previous day's contaminant (\ce{AlEt3}) to initiate the reaction.
        \item Present-day \emph{super-active} or \emph{high-mileage} system: A heterogeneous catalyst.
        \begin{itemize}
            \item Ball milling...
        \end{itemize}
        \item With \ce{MgCl2} support, the catalyst produces as much as 7 kg PE/hr.
        \item Ziegler won for polyethylene polymerization; Natta realized that it could be done with polypropylene.
    \end{itemize}
    \item Reaction of the Ziegler-Natta components.
    \begin{itemize}
        \item Two different models.
        \item Important things to remember: Unoccupied/vacant orbital at the catalyst site, coordination to the monomer.
    \end{itemize}
    \item Tacticity in polymers.
    \begin{itemize}
        \item Isotactic: Chain ends must add to the same face of every double bond.
        \item Syndiotactic: Hitting opposite faces in succession.
    \end{itemize}
    \item Mechanistics.
    \begin{itemize}
        \item We believe that tacticity is controlled by the nature of the catalytic site: So this is a catalyst site to control mechanism.
        \item Thus, for isotactic addition, we need a chiral catalyst site.
        \item Let's look at the surface of this heterogeneous (solid) catalyst and the gaseous monomer it reacts with.
        \item Titanium likes to be hexacoordinate.
        \item The way the titaniums bond to each other makes some have $D_3$/paddlewheel symmetry in a sense, and others be enantiotopic to that.
        \item Broad MWD is usually observed for heterogeneous Ziegler-Natta systems (5-30 have been reported).
        \item Equivalent homogeneous system uses vanadium. This does syndiotactic PP.
        \begin{itemize}
            \item Rowan explains the alternating growth mechanism from CHEM20200Notes.
            \item The metal will react with the carbon bearing the R group in the monomer.
        \end{itemize}
    \end{itemize}
    \item Great summary of homogeneous vs. heterogeneous.
    \begin{itemize}
        \item ...
    \end{itemize}
    \item Termination of chain growth.
    \begin{itemize}
        \item ...
    \end{itemize}
    \item Polymerization of dienes with Ziegler-Natta catalysts.
    \begin{itemize}
        \item Selective polymerization of butadiene, for instance.
        \item Four different stereoregular polymers...
    \end{itemize}
    \item \textbf{Metallocene}: A positive metal ion sandwiched between two negatively charged cyclopentadienyl anions.
    \begin{itemize}
        \item Ziegler-Natta was the first generation.
        \item Metallocene advantages.
        \begin{enumerate}
            \item Over 100-fold more reactive than heterogeneous Z-N initiators.
            \item Very high molecular weight polymers.
            \item Single site initiators.
            \begin{itemize}
                \item Polymer with better stereo- and regio-control.
                \item Narrower molecular weight.
            \end{itemize}
        \end{enumerate}
        \item Zr is bigger than Fe, thus better and more accessible active site.
        \item The indenyl anion is another aromatic anion; binding two of these together with an \textbf{ethylene bridge} forms a bulky ligand that can be used to help direct the monomers to react in the correct orientation.
        \begin{itemize}
            \item These produce isotactic PP very well.
        \end{itemize}
        \item We activate the catalyst by ripping off the chlorines and alkylating.
    \end{itemize}
    \item Mechanism.
    \begin{itemize}
        \item Polymerization alternates between the two different ligand positions.
        \item The double bond is inserted between the metal and the polymer through a 4-membered transition state.
        \item ...
    \end{itemize}
    \item If you could force the polymer to attack the opposite faces of the olefin alternatively, then you can produce syndiotactic polymers.
    \begin{itemize}
        \item Another catalyst does this!
    \end{itemize}
    \item Oscillating metallocenes.
    \begin{itemize}
        \item Without the tethering group, the aromatic anions rotate.
        \item But they rotate relatively slowly between their conformers, so we get stereoblock copolymers.
    \end{itemize}
    \item Altogether, metallocenes have active sites that are fairly constrained, which is good for controlling tacticity but not good for bigger monomers.
    \item \emph{ansa}-Cyclopentadienyl-amido initiators.
    \begin{itemize}
        \item ...
    \end{itemize}
    \item Phenoxy-imine chelate initiators.
    \begin{itemize}
        \item \ce{R^1} is usually phenyl or substituted...
    \end{itemize}
    \item Issue: Most of these catalysts are very oxo-philic, so $\alpha$-olefins work, but basically anything else doesn't.
    \item Potential solution: $\alpha$-diimine chelates of late transition metals.
    \begin{itemize}
        \item Late transition metals (e.g., \ce{Ni} and \ce{Pd}) are less oxophilic. However, such systems tend to have low activity (forming only oligomers) on account of extensive $\beta$-hydride transfer.
        \item Propagation and branching mechanism.
    \end{itemize}
    \item We now move into olefin metathesis.
    \item Olefin metathesis (ROMP, ADMET).
    \begin{itemize}
        \item We're essentially breaking two old double-bonds and forming two new double bonds.
    \end{itemize}
    \item Catalysts.
    \begin{itemize}
        \item Ziegler-Natta-type initiators can be good.
        \item We have a four-membered transition state (hotly contested, but eventually proven and later modeled by Yves Chauvin --- one of the 2005 Nobel Prize people, along with Schrock and Grubbs).
        \item Schrock, Grubbs I, and Grubbs II covered.
        \begin{itemize}
            \item Nowadays, Aldrich sells specialized versions of these that have been tuned and tailored for various applications.
        \end{itemize}
    \end{itemize}
    \item Mechanism for how these catalysts work.
    \begin{itemize}
        \item ...
    \end{itemize}
    \item Commercial olefin metathesis polymers.
    \begin{itemize}
        \item Story about dicyclopentadiene from Bob Grubbs.
        \begin{itemize}
            \item Materia was the company.
            \item It was a very durable polymer.
            \item He got baseball players to fund it by making a baseball bat and then one decided to bash it against the ground and it didn't break, so they all ventured in.
        \end{itemize}
        \item Rate of polymerization vs. rate of backbiting.
        \item You have to stop the reaction very quickly.
        \item You can graft things off the side of polymers.
    \end{itemize}
    \item ROMP is by far the dominant method of olefin metathesis, but it's not the only one.
    \item Acyclic Diene Metathesis Polymerization (ADMET)\footnote{Sounds like Anderson misnamed this!}.
    \begin{itemize}
        \item This is a condensation polymerization.
    \end{itemize}
\end{itemize}



\section{Copolymerization}
\begin{itemize}
    \item \marginnote{2/15:}Copolymerization why does this matter?
    \begin{itemize}
        \item Copolymers are polymers made from more than one monomer.
        \item Wide range of property profiles can be achieved from only a few monomers\dots by making block, graft, or statistical copolymers.
        \item This section...
    \end{itemize}
    \item Copolymerization.
    \begin{itemize}
        \item Polymerization of more than one monomer at the same time.
        \item You almost never get 50/50, though, because one monomer will be more reactive than the other (or because they are introduced in different concentrations).
        \item Step-growth copolymerization: Many step-growth polymers are copolymers. As the polymerization is carried out close to 100\% (for high MWt), then the overall composition of the polymer is usually the same as the feed composition.
        \item Chain copolymerization: As (co)polymers are formed throughout the polymerization then copolymerization is more complicated. This is the focus on this section.
    \end{itemize}
    \item Chain copolymerization.
    \begin{itemize}
        \item For chain polymerization, the polymer chemical microstructure (i.e., the amount and placement of the different monomers) depends on \emph{both} relative monomer concnetrations and relative reactivities. Note that depending on the reactivity of the monomers, the monomer feed can change during the reaction.
        \item Now there's \textbf{composition drift}, too.
        \item At low conversions, the more reactive monomer will be incorporated more often. At high conversions, the less reactive monomer will be more left over and its excess of concentration means that it will start to be incorporated more.
    \end{itemize}
    \item Types of copolymers made by copolymerization of two monomers.
    \begin{itemize}
        \item Statistical and random copolymers: A copolymer consisting of macromolecules in which the sequential distribution of the monomeric units obeys known statistical laws.
        \begin{itemize}
            \item Note: An example of a statistical copolymer is one consisting of macromolecules in which the sequential distribution of monomeric units follows Markovian statistics.
            \item Truly random copolymers are formed if the probability of finding a given type monomer residue at a particular point in the chain is equal to the mole fraction of that monomer resideu in the chain (Bernoullian [zero-order Markov]).
            \item All randoms are statistical; not all statistical are random.
        \end{itemize}
        \item Alternating copolymer: An alternating copolymer is a copolymer comprising two species of monomeric...
    \end{itemize}
    \item What determines microstructure?
    \begin{itemize}
        \item ...
        \item Reactivit of monomers in copolymerization cannot be determined by the knowledge of the homopolymerizatios of both monomers.
        \item Some monomer sare more reactive in copolymerization than would be indicated by their rate of homopolymerization.
        \begin{itemize}
            \item Some monomers that cannot polymerize \emph{at all} on their own polymerize beautifully with others.
            \item Think bulky monomers that can't react with each other but can be strung together by smaller molecules.
        \end{itemize}
        \item Some monomers that cannot...
        \item \textbf{First-order Markov} or \textbf{terminal model} of copolymerization: Assumes that the chemical reactivity of the propagating chain depends only on the nature of the monomer units at the chain end (and is independent of what $P$ is).
        \item There is \textbf{self-propagation} and \textbf{cross-propagation}.
        \begin{itemize}
            \item Monomers (at growing chain ends) reacting with like monomers vs. monomers (at growing chain ends) reacting with unlike monomers.
        \end{itemize}
        \item Assumptions (to make things not ridiculously complicated).
        \begin{itemize}
            \item ...
        \end{itemize}
        \item You can take this a step further by considering the effect of the next to last repeat unit. This is called the \textbf{penultimate} control mechanism.
        \begin{itemize}
            \item Good to know and be aware of, but there are a lot more equations.
            \item So in this course, we will focus only on the terminal model.
        \end{itemize}
    \end{itemize}
    \item Copolymerization equation derivation.
    \begin{itemize}
        \item Goal: Find a way to express the conversion of \ce{M_1} and \ce{M_2} into the polymer over time.
        \item Recall the four propagation equations.
        \begin{align*}
            \ce{PM_1* + M_1} &\ce{->[$k_{11}$]} \ce{PM_1M_1*}\tag*{$R_{p,11}=k_{11}[\ce{PM_1*}][\ce{M_1*}]$}
            &...
        \end{align*}
        \item This means that the rate of \ce{M_1} consumption (rate of its addition into the polymer) is given by
        \begin{equation*}
            \dv{[\ce{M_1}]}{t} = R_{p,11}+R_{p,21} = ...
        \end{equation*}
        and
        \begin{equation*}
            ...
        \end{equation*}
        \item It follows that
        \begin{equation*}
            ...
        \end{equation*}
        \item If we make a quasi-steady state assumption that the total concentration of radicals is constant, then the rate of cross-over between different types of terminal units is equal. Mathematically,
        \begin{equation*}
            \frac{[\ce{PM_1*}]}{[\ce{PM_2*}]} = \frac{k_{21}[\ce{M_1}]}{k_{12}[\ce{M_2}]}
        \end{equation*}
        \item Combining the previous two equations, we obtain
        \begin{equation*}
            \frac{\dv*{[\ce{M1}]}{t}}{\dv{[\ce{M2}]}{t}} = \frac{[\ce{M1}]}{[\ce{M2}]}\cdot\frac{(k_{11}/k_{12})[\ce{M1}]+[\ce{M2}]}{(k_{22}/k_{21})[\ce{M2}]+[\ce{M1}]}
        \end{equation*}
        \begin{itemize}
            \item We can measure monomer concentrations over time using NMR or something, so this allows us to get the rate constant ratios in the above equation.
        \end{itemize}
        \item These rate constant ratios are formally known as \textbf{reactivity ratios} and tell you the ratio of self-reactivity compared to cross-reactivity.
        \begin{itemize}
            \item For binary systems, $r_{ij}$ may be shortened to $r_i$ and $r_{ji}$ may be shortened to $r_j$.
            \item Thus,
            \begin{align*}
                r_1 &:= \frac{k_{11}}{k_{12}}&
                r_2 &:= \frac{k_{22}}{k_{21}}
            \end{align*}
            \item Example: If $r_1<1$, then \ce{PM_1*} preferentially adds to \ce{M2}.
            \item If $r_1>1$, then \ce{PM_1*} preferentially adds to \ce{M1}
            \item If $r_1=0$, then that monomer cannot homopolymerize.
            \item Note that reactivity ratios --- like rate constants --- depend on temperature, solvent, etc.
        \end{itemize}
    \end{itemize}
    \item Composition vs. feedstock.
    \begin{itemize}
        \item All of the previous equation was based on concentration.
        \item It can be useful, instead, to describe the \emph{mole fraction} of the monomer/repeat unit in both the feedstock and polymer (instead of concentrations).
        \item $F_1$ is the mole fraction of \ce{M1} in the polymer.
        \begin{itemize}
            \item This means that $F_1+F_2=1$.
        \end{itemize}
        \item $f_1$ is the mole fraction of \ce{M1} in the feed.
        \begin{itemize}
            \item This means that $f_1+f_2=1$.
        \end{itemize}
        \item Thus,
        \begin{align*}
            F_1 &= 1-F_2
                = \frac{\dv*{[\ce{M1}]}{t}}{\dv*{[\ce{M1}]}{t}+\dv*{[\ce{M2}]}{t}}&
            f_1 &= 1-f_2
                = \frac{[\ce{M1}]}{[\ce{M1}]+[\ce{M2}]}
        \end{align*}
        \item Combining these equations with the previous, concentration-based result, we get
        \begin{align*}
            F_1 &= \frac{r_1f_1^2+f_1f_2}{r_1f_1^2+2f_1f_2+r_2f_2^2}&
            \frac{F_1}{F_2} &= ...
        \end{align*}
        \begin{itemize}
            \item Either of these is referred to as the \textbf{Mayo Lewis equation}.
        \end{itemize}
    \end{itemize}
    \item Microstructure of copolymers.
    \begin{itemize}
        \item The copolymer composition equation describes the overall composition of the copolymer but does not say anythign about exact arrangement of the two monomers along the polymer chain.
        \item It is important to note that unless $r_1=r_2=1$, the placement is \emph{not} random and should properly be considered \emph{statistical}.
        \item The microstructure of the copolymer is defined by the distribution of the various lengths of \ce{M1} and \ce{M2} sequences, the \textbf{sequence length distribution}.
        \item The mole fractions of $(N_1)_x$ and $(N_2)_x$ of forming \ce{M1} and \ce{M2} sequences of length $x$ are...
        \begin{itemize}
            \item No need to focus on this mathematically.
        \end{itemize}
        \item Visually, however, the equations tell us that if $r_1=r_2=1$ (i.e., $f_1=f_2$), then the probability of finding repeats of length $x$ decreases exponentially.
        \begin{itemize}
            \item In cases where $r_1$ is big and $r_2$ is small, most \ce{M2}'s will be by themself and there will be a more stretched out exponential distribution of \ce{M1}.
            \item In cases where both are small, both \ce{M1} and \ce{M2} will very much most commonly appear by themselves. This gives us an \textbf{alternating copolymer}.
        \end{itemize}
    \end{itemize}
    \item Reactivity ratios.
    \begin{itemize}
        \item A few notes.
        \item Get comfortable with this slide by any means necessary!!
        \begin{itemize}
            \item "Learn it, study it, climb into bed and sleep with it if you have to."
            \item This slide makes everything in this presentation make sense.
        \end{itemize}
        \item Repeats some stuff from above.
        \item Both ratios are needed to characterize the system: Although $r_1$ is descriptive of the radical \ce{M1*}, it also depends on the identity of \ce{M2*}; ...
        \item ...
    \end{itemize}
    \item Determining reactivity ratios.
    \begin{itemize}
        \item We can use the mole fraction equation.
        \item We can control the monomer feed ratio $f_1,f_2$, so the key is then to measure the copolymer composition $F_1,F_2$.
        \item To get $F$, we use analytical techniques that allow us to determine the amount of each repeat unit.
        \begin{itemize}
            \item Examples: Elemental analysis, NMR, MS, FT-IR, UV, etc.
            \item Remember, it is important to properly purify up the sample to remove solvent, initiators, monomers, etc.
        \end{itemize}
        \item Back in the day, we could use the Mayo-Lewis method, but it was wildly inaccurate; no one does it any more.
        \item We use a lot of nonlinear methods and curve fitting.
        \item Look at sterics in the example!
    \end{itemize}
    \item Reactivity ratios.
    \begin{itemize}
        \item There are some tables in books of reactivity ratios and the products for selected copolymers at given temperatues.
        \item There aren't many examples where both $r_1,r_2>1$; thus, for block-copolymers, we probably need a different method.
        \item Note that $0<r_1r_2<1$ in general.
    \end{itemize}
    \item Type of copolymerization behavior.
    \begin{itemize}
        \item When $r_1r_2=1$, we call this an \textbf{ideal copolymerization}.
        \begin{itemize}
            \item This is when the propagating species, \ce{PM1*} or \ce{PM2*}, show the same preference for adding one or the other of the two monomers, i.e.,
            \begin{equation*}
                \frac{k_{22}}{k_{21}} = \frac{k_{12}}{k_{11}}
                \qquad\text{or}\qquad
                r_2 = \frac{1}{r_1}
            \end{equation*}
            and the relative rates of incorporation of the two monomers into the copolymer are independent of the propagating species. Using prior copolymerization equations for an ideal copolymerization, we have
            ...
            \item Note...
        \end{itemize}
        \item A special case of ideal copolymerization: When $r_1=r_2=1$.
        \begin{itemize}
            \item Here, the copolymer composition is the same as the monomer feed.
            \item We can calculate that $F_1=f_1$ from the equation.
            \item This is true \textbf{random} or \textbf{Bernoullian} behavior.
        \end{itemize}
        \item In an ideal copolymerization when $r_1\neq r_2$\dots
        \begin{itemize}
            \item As $r_1$ increases, then $F_1$ increases.
            \item It becomes harder to produce copolymers containign both monomers as the difference in $r_1,r_2$ increases.
            \item Only if $r_1,r_2$ are not too different, e.g., $r_1$ is 0.5-2 can polymers be obtained with an appreciable amount of both monomers.
        \end{itemize}
        \item If $r_1r_2=0$\dots
        \begin{itemize}
            \item If, in addition, neither $r_1$ nor $r_2$ is greater than 1, then the propagating species prefers to react with the other monomer. This leads to
            \begin{equation*}
                ...
            \end{equation*}
            \begin{itemize}
                \item Examples are rare, but they do exist.
                \item Example: Radical polymerization of stilbene and maleic anhydride.
            \end{itemize}
        \end{itemize}
        \item Block copolymers.
        \begin{itemize}
            \item Even rarer.
            \item The example given is very hard to do.
        \end{itemize}
        \item Most copolymerizations are $0<r_1r_2<1$.
        \begin{itemize}
            \item The tendency toward alternation and the tendency away fron ideal behavior increases as $r_1r_2$ moves from 1 to 0.
            \item For the cases where both $r_1,r_1<1$, ...
        \end{itemize}
    \end{itemize}
    \item Radical olefin copolymerization.
    \begin{itemize}
        \item Reactivity ratios are kinetic in origin and are therefore a reflection of transition state of the reaction.
        \item Effect of reaction conditions.
        \begin{itemize}
            \item Reaction medium: Generally, solvent does not have a big effect on reactivity ratios.
            \item Temperature: If the polymerization is irreversible, then temperature does not have a large effect on reactivity ratios in radical copolymerization.
            \item The variation with $T$ will depend on the difference in poropagation activation energies.
            \item As the activatino energies of radical propagation are small and similar...
        \end{itemize}
    \end{itemize}
    \item Relation of reactivity to chemical structure.
    \begin{itemize}
        \item The reactivity ratios can be...
        \item We can examine the reactivities of the radicals and monomers by looking at the rates of cross-propagation (i.e., the rate of radical 1 reacting with monomer 2, which is $k_{12}$).
        \item Values of cross-propagation constants $k_{12}$ (\si{\liter\per\mole\per\second}) for four monomer-radical combinations.
        \item Resonance stabilization has more of an effect on the reactivity of the radical than the reactivity of the monomer.
        \item Resonance: Consider species that either have or have not resonance stabilization (rs).
        \item A stabilized radical reacts with a stabilized monomer to form a stabilized radical.
        \item ...
        \item The last one is by far the most favorable reaction.
        \item Steric effects --- the counterpart to rs.
        \item The \emph{cis} option is super destabilized.
    \end{itemize}
\end{itemize}




\end{document}